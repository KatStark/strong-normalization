\documentclass[preprint]{sigplanconf}
%\documentclass{sigplanconf}
\usepackage{color}
\usepackage{xspace}
\usepackage{listings}

% ---------------------------------------------------------------------------
% ------------------------------ Contextual ML ------------------------------
% ---------------------------------------------------------------------------

\lstdefinelanguage{ContextualML}
{
  morekeywords={and, block, case, of, mlam, fn, impossible, let, in, schema,
    some, rec, type, ctype, prop, stratified, inductive, coinductive, LF, if, then,
    else, total},
  keepspaces=true,
  sensitive,
  morecomment=[l]{\%},
  morecomment=[n]{\%\{}{\}\%},
  morestring=[b]"
}[keywords,comments,strings]

\lstloadlanguages{ContextualML}
\lstset{language=ContextualML}

\newdimen\zzlistingsize
\newdimen\zzlistingsizedefault
\zzlistingsizedefault=10pt
\zzlistingsize=\zzlistingsizedefault
\global\def\CommentCopter{0}
\newcommand{\Lstbasicstyle}{\fontsize{\zzlistingsize}{1.05\zzlistingsize}\ttfamily%
}
\newcommand{\keywordcopter}{\fontsize{0.95\zzlistingsize}{1.0\zzlistingsize}\bf}
\newcommand{\stupidcopter}{\if0\CommentCopter\keywordcopter\fi}
\newcommand{\commentcopter}{\def\CommentCopter{1}\fontsize{0.95\zzlistingsize}{1.0\zzlistingsize}\rmfamily\slshape}

\newcommand{\caret}{\char94}

\newcommand{\LST}{\setlistingsize{\zzlistingsizedefault}}

\newlength{\zzlstwidth}
\newcommand{\setlistingsize}[1]{\zzlistingsize=#1%
\settowidth{\zzlstwidth}{{\Lstbasicstyle~}}%
%\setlength{\zzlstwidth}{3pt}%
}
\setlistingsize{\zzlistingsizedefault}

% The order of the "literate" definitions is significant:
%   later definitions shadow earlier ones.  The \\Pi definition must come
%   *after* the \\ definition, or the first part of \\Pi --- that is, \\ --- will
%   be matched, and instead of $\Pi$ you'll get $\lambda Pi$.
%
\lstset{literate={->}{{$\rightarrow~$}}2 %
                 {=>}{{$\Rightarrow~$}}2 %
                 {|-}{{$\vdash\,$}}2 %
                 {..}{{$.\hspace{-0.025cm}.\hspace{-0.025cm}.$}}1 % is there any nicer way?
                 {\\}{{$\lambda$}}1 %
                 {\\Pi}{{$\Pi$}}1 %
                 {\\gamma}{{$\gamma$}}1 %
                 {\\psi}{{$\psi$}}1 %
                 {\\sigma}{{$\sigma$}}1 %
                 {FN}{{$\Lambda$}}1 %
                 {<<}{\color{ForestGreen}}1 %
                 {<<r}{\color{FireBrick}}1 %
                 {<*}{\color{ForestGreen}}1 %
                 {<dim}{\color{DimGrey}}1 %
                 {>>}{\color{black}}1 %
                 {?}{\bf{?}}1,
        columns=[l]fullflexible,
        basicstyle=\ttfamily\lst@ifdisplaystyle\footnotesize\fi,
        keywordstyle=\bf,
        identifierstyle=\relax,
        stringstyle=\relax,
        commentstyle=\slshape\color{DimGrey},
        breaklines=true,
        % breakatwhitespace=true,   % doesn't do anything (?!)
        mathescape=true,   % interprets $...$ in listing as math mode
        xleftmargin=0.5cm,
      }


% ---------------------------------------------------------------------------

% %                               {^}{{$\caret$}}1 %
% \lstset{literate=%
% %                               {=>}{{$\Rightarrow~$}}2 %
% %                               {==>}{{$\Longrightarrow~$}}3 %
%                                {\\/}{{$\unty$}}2 %
% %                               {-}{-}1 %
% %                               {->}{{$\rightarrow~$}}2 %
%                                {^}{{$\,\caret\,$}}1 %
%                                {|-}{{$\vdash$}}1 %
%                                {orelse}{{\stupidcopter orelse}$~$}3 %
%                                {as}{{\stupidcopter as}$~$}1 %
%                                {else}{{\stupidcopter else}$~$}3 %
%                                {case}{{\stupidcopter case}$~$}3 %
%                                {mlam}{{\stupidcopter mlam}$~$}3 %
%                                {raise}{{\stupidcopter raise}$~$}4 %
%                                {let}{{\stupidcopter let}$~$}2 %
%                                {datatype}{{\stupidcopter datatype}$~$}6 %
%                                {datasort}{{\stupidcopter datasort}$~$}6 %
%                                {@@@}{\text{\$}}1 %
%                ,
%                columns=[l]fullflexible,
%                basewidth=\zzlstwidth,
%                basicstyle=\Lstbasicstyle,
%                keywordstyle=\keywordcopter,
%                identifierstyle=\relax,
% %               stringstyle=\relax,
%                commentstyle=\commentcopter,
% %               prebreak={\mbox{$\space\swarrow$}},
% %               postbreak={\mbox{$\space\searrow$}},
%                showstringspaces=false,
%                breaklines=true,
%                breakatwhitespace=true,
%                mathescape=true,
% %               tabsize=8,
%                texcl=false}

% \setlistingsize{11pt}

% \usepackage{pstricks,pst-node,pst-tree}
\usepackage{graphics}
\usepackage{graphicx}

\newdimen\zzfontsz
\newcommand{\fontsz}[2]{\zzfontsz=#1%
{\fontsize{\zzfontsz}{1.2\zzfontsz}\selectfont{#2}}}

\newcommand{\mathsz}[2]{\text{\fontsz{#1}{$#2$}}}

\newtheorem{definition}{Definition}[section]
\newtheorem{theorem}{Theorem}[section]
\newtheorem{conjecture}[theorem]{Conjecture}
\newtheorem{corollary}[theorem]{Corollary}
\newtheorem{proposition}[theorem]{Proposition}
\newtheorem{lemma}[theorem]{Lemma}

\renewcommand{\Tilde}{\textsf{\char"7E}}

\newcommand{\arrayenv}[1]{\renewcommand{\arraystretch}{1} \begin{array}[t]{@{}c@{}}#1\end{array}}
\newcommand{\arrayenvc}[1]{\renewcommand{\arraystretch}{1} \begin{array}[c]{@{}c@{}}#1\end{array}}
\newcommand{\arrayenvr}[1]{\renewcommand{\arraystretch}{1} \begin{array}[t]{@{}r@{}}#1\end{array}}
\newcommand{\arrayenvbr}[1]{\renewcommand{\arraystretch}{1} \begin{array}[b]{@{}r@{}}#1\end{array}}
\newcommand{\arrayenvl}[1]{\renewcommand{\arraystretch}{1} \begin{array}[t]{@{}l@{}}#1\end{array}}
\newcommand{\arrayenvb}[1]{\renewcommand{\arraystretch}{1}  \begin{array}[b]{@{}c@{}}#1\end{array}} 
\newcommand{\arrayenvbl}[1]{\renewcommand{\arraystretch}{1}  \begin{array}[b]{@{}l@{}}#1\end{array}}

\newcommand{\subtype}{\leq}

\newcommand{\union}{\mathrel{\cup}}
\newcommand{\sect}{\mathrel{\cap}}

\newcommand{\unit}{\texttt{()}}
\newcommand{\Unit}{\textsf{unit}}
\newcommand{\bang}{\texttt{!}}
\renewcommand{\gets}{\mathop{\texttt{:=}}}

\newcommand{\down}{\mathrel{\,\Downarrow\,}}
\newcommand{\step}{\mathrel{\,\Rightarrow\,}}
\newcommand{\mstep}{\longrightarrow^*}

\newcommand{\D}{\mathcal{D}}
\newcommand{\E}{\mathcal{E}}
\newcommand{\F}{\mathcal{F}}



\newcommand{\Rsectintro}{\textsc{$\sectty$intro}\xspace}
\newcommand{\Rsectelim}[1]{\textsc{$\sectty$elim{#1}}\xspace}

\newcommand{\TIf}{\textsc{t-if}}
\newcommand{\TPlus}{\textsc{t-plus}}
\newcommand{\TMult}{\textsc{t-mult}}
\newcommand{\TEq}{\textsc{t-eq}}
\newcommand{\TApp}{\textsc{t-app}}
\newcommand{\TSub}{\textsc{t-sub}}
\newcommand{\TFn}{\textsc{t-fn}}
\newcommand{\TFun}{\textsc{t-fun}}
\newcommand{\TPair}{\textsc{t-pair}}
\newcommand{\TFst}{\textsc{t-fst}}
\newcommand{\TSnd}{\textsc{t-snd}}
\newcommand{\TVar}{\textsc{t-var}}
\newcommand{\TNum}{\textsc{t-num}}
\newcommand{\TTrue}{\textsc{t-true}}
\newcommand{\TFalse}{\textsc{t-false}}

\newcommand{\TBinaryPrimop}{\textsc{t-binary-primop}\xspace}
\newcommand{\TUnaryPrimop}{\textsc{t-unary-primop}\xspace}
\newcommand{\TTuple}{\textsc{t-tuple}\xspace}
\newcommand{\TTupleSyn}{\textsc{t-tuple-syn}\xspace}
\newcommand{\TRec}{\textsc{t-rec}\xspace}
\newcommand{\TAnno}{\textsc{t-anno}\xspace}

\newcommand{\TLet}{\textsc{t-let}\xspace}
\newcommand{\TLetSyn}{\textsc{t-let-syn}\xspace}
\newcommand{\TDecs}{\textsc{t-decs}\xspace}
\newcommand{\TByName}{\textsc{t-by-name}}
\newcommand{\TByVal}{\textsc{t-by-val}}
\newcommand{\TByValTuple}{\textsc{t-by-val-tuple}}

\newcommand{\SIFT}{\textsc{s-iftrue}\xspace}
\newcommand{\SIFF}{\textsc{s-iffalse}\xspace}
\newcommand{\SIF}{\textsc{s-if}\xspace}
\newcommand{\SAppFnStep}{\textsc{s-app-fn}\xspace}
\newcommand{\SAppArgStep}{\textsc{s-app-arg}\xspace}
\newcommand{\SAppBeta}{\textsc{s-app}\xspace}

\newcommand{\BAnno}{\textsc{b-anno}\xspace}
\newcommand{\BAnnoFn}{\textsc{b-anno-fn}\xspace}
\newcommand{\BAnnoNonFn}{\textsc{b-anno-non-fn}\xspace}
\newcommand{\BIFT}{\textsc{b-iftrue}\xspace}
\newcommand{\BIFF}{\textsc{b-iffalse}\xspace}
\newcommand{\BOp}{\textsc{b-op}\xspace}
\newcommand{\BPlus}{\textsc{b-plus}\xspace}
\newcommand{\BEq}{\textsc{b-eq}\xspace}
\newcommand{\BLet}{\textsc{b-let}\xspace}
\newcommand{\BNum}{\textsc{b-num}\xspace}
\newcommand{\BVar}{\textsc{b-var}\xspace}
\newcommand{\BIF}{\textsc{b-if}\xspace}
\newcommand{\BTrue}{\textsc{b-true}\xspace}
\newcommand{\BFalse}{\textsc{b-false}\xspace}
\newcommand{\BFun}{\textsc{b-fun}\xspace}
\newcommand{\BRec}{\textsc{b-rec}\xspace}

\newcommand{\base}{\textsf{i}}
\newcommand{\Int}{\textsf{int}}
\newcommand{\Float}{\textsf{float}}
\newcommand{\Bool}{\textsf{bool}}
\newcommand{\Real}{\textsf{real}}
\newcommand{\String}{\textsf{string}}
\newcommand{\Char}{\textsf{char}}
\newcommand{\Ref}{~\textsf{ref}}
\newcommand{\Array}{~\textsf{array}}
\newcommand{\norm}{\mathrel{\,\uparrow\,}}
\newcommand{\neut}{\mathrel{\,\downarrow\,}}
\newcommand{\neutG}{\Gamma^{\downarrow}}
\newcommand{\syn}{\mathrel{\,\Rightarrow\,}}
\newcommand{\chk}{\mathrel{\,\Leftarrow\,}}
\newcommand{\arr}{\mathrel{\texttt{->}}}
% \newcommand{\arrow}{\arr}
\newcommand{\entails}{\vdash}
\newcommand{\such}{~|~}
\newcommand{\sectty}{\mathrel{\text{\&}}}

\newcommand{\tmtrue}{\textsf{true}}
\newcommand{\tmfalse}{\textsf{false}}
\newcommand{\tmif}[3]{\textsf{if\;} #1 \textsf{\;then\;} #2 \textsf{\;else\;} #3}
\newcommand{\tmfun}[3]{\textsf{fun } #1 (#2) = #3}
\newcommand{\tmfn}[2]{\textsf{fn } #1\;\texttt{=>}\;#2}
\newcommand{\tmapp}[2]{#1\;#2}
\newcommand{\tmrec}[3]{\textsf{rec } {#1}\,:\,{#2}\;\texttt{=>}\;#3}
\newcommand{\tmlet}[3]{\textsf{let } #1 = #2 \textsf{\;in\;} #3\; \textsf{end}}

 \newcommand{\tmfst}[1]{\textsf{fst}\;{#1}\xspace}
 \newcommand{\tmsnd}[1]{\textsf{snd}\;{#1}\xspace}

% Numerical expressions
\newcommand{\tmzero}{\textsf{z}}
\newcommand{\tmsucc}{\textsf{succ}}
\newcommand{\tmpred}{\textsf{pred}}
\newcommand{\tmiszero}{\textsf{iszero}}


\newcommand{\BLetn}{\textsc{b-letn}\xspace}
\newcommand{\BLetp}{\textsc{b-letpair}\xspace}
\newcommand{\BPair}{\textsc{b-pair}\xspace}
\newcommand{\BFst}{\textsc{b-fst}\xspace}
\newcommand{\BSnd}{\textsc{b-snd}\xspace}
\newcommand{\BFn}{\textsc{b-fn}\xspace}
\newcommand{\BApp}{\textsc{b-app}\xspace}

\newcommand{\unif}{\doteq}
\newcommand{\totp}{\Rightarrow}
\newcommand{\emp}{\emptyset}
\newcommand{\TT}{\textsf{tt}}

% \newcommand{\D}{{\cal{D}}}
\newcommand{\FV}{\mathsf{FV}}

% \newcommand{\m}[1]{\mbox{\lstinline!#1!}}
\newcommand{\mlam}[2]{\m{mlam}\; #1 \Rightarrow #2}
\newcommand{\fn}[2]{\m{fn}\;#1 \Rightarrow #2}
\newcommand{\rec}[2]{\m{rec}\;#1 = #2}
\newcommand{\boxm}[2]{\m{box}(#1.\,#2)}
\newcommand{\letboxm}[3]{\m{let}\,[#1]\;{=}\;#2\;\m{in}\;#3}
\newcommand{\casev}[2]{\m{case}\;#1\;\m{of}\;#2}
\newcommand{\casebox}[2]{\casev{#1}{#2}}
\newcommand{\casearm}[2]{#1 \Rightarrow #2}
\newcommand{\branch}[4] {\Pibox #1.#2:#3 \Rightarrow #4}



%\usepackage{graphics} % use for the Pitts-Gabbay quantifier
%\usepackage{color}
% \input listing-macros
% \usepackage{todonotes}
% \newcommand{\inlinetodo}[1]{\todo[inline,color=green!40]{#1}}
% \newcommand{\inlinetodoam}[1]{\todo[inline,color=red!40]{#1 -- Alberto}}
% \newcommand{\inlinetodoaa}[1]{\todo[inline,color=green!40]{#1 -- Andreas}}
% \newcommand{\inlinetodobp}[1]{\todo[inline,color=yellow!40]{#1 -- Brigitte}}
%\newcommand{\inlinetodo}[1]{\ednote {#1}} 

%\long\def\ednote#1{\footnote{[{\it #1\/}]}\message{ednote!}}
% \long\def\note#1{\begin{quote}[{\it #1\/}]\end{quote}\message{note!}}

%%%%%%%%%%%%%%%%%%%%%
\newcommand{\sn}[2]{#1 \vdash #2 \in \ensuremath{\mathsf{SN}}}
% \newcommand{\rcs}[2]{\ensuremath{\Delta\vdash\mathcal{R}(#1)\hastype
% #2 }} % reducibility canadidate
\newcommand{\rcs}[3]{\ensuremath{#1 \vdash #2\in \mathcal{R}_{#3} }} % reducibility canadidate



\begin{document}

\conferenceinfo{}{}
\CopyrightYear{}
\copyrightdata{}

\title{POPLMark Reloaded}
\authorinfo
 {Andreas Abel}
 {Department of Computer Science and Engineering, Gothenburg University / Chalmers, Sweden}
 {andreas.abel@gu.se}

 \authorinfo
 {Alberto Momigliano}
 {DI, Universit\`a degli Studi di Milano, Italy }{momigliano@di.unimi.it}
\authorinfo
 {Brigitte Pientka}
 {School of Computer Science, McGill University, Montreal,
           Canada}
 {bpientka@cs.mcgill.ca}
\maketitle

\begin{abstract}
  As a follow-up to the POPLMark Challenge, we propose a new benchmark
  for machine-checked metatheory of programming languages:
  establishing strong normalization of a simply-typed lambda-calculus
  with a proof by Kripke-style logical relations. We believe that this
  case-study overcomes some of the limitations of the original
  challenge and highlights, among others, the need of native support
  for context reasoning and simultaneous substitutions.
\end{abstract}


\category{D.3.1}{Programming Languages}{Formal Definitions and Theory}
\category{F.3.1}
         {Logics and Meanings of Programs}
         {Specifying and Verifying and Reasoning about Programs}
\category{F.4.1}{Mathematical Logic}{Lambda Calculus and Related
  Systems}[Mechanical theorem proving, Proof theory]
% \category{I.2.3}{Artificial Intelligence}{Deduction and Theorem
%   Proving}[Deduction, Inference engines, meta
% theory] \terms{Theory, Languages, Verification}



\keywords
Machine checked meta-theory,
benchmarks,
POPLMark Challenge,
logical frameworks,
strong normalization,
logical relations


\section{Introduction}
\label{sec:intro}
The usefulness of sets of benchmarks has  been recognized in many areas
of computer science, and in particular in the theorem proving
community, for stimulating progress or at least taking stocks of what
the state of the art is --- \emph{TPTP}~\citep{TPTP} is one  shining
example. The situation is less satisfactory for proof assistants,
where each system comes with its own set of examples/libraries, some
of them gigantic; this is  not surprisingly, since we are
potentially addressing the whole realm of mathematics.



In a more limited setting, some 12 years ago, a group of renowned
programming language theorists came together and issued the so-called
\emph{POPLMark Challenge}~\citep{Aydemir05TPHOLs} (PC, in short), with
the aim of fostering the collaboration between the PL community and
researchers in proofs assistants/logical frameworks to bring about:
\begin{quote}
  ``[\dots] a future where the papers in conferences such as POPL and
  ICFP are routinely accompanied by mechanically checkable proofs of
  the theorems they claim'' (page 51 op.\ cit.)
\end{quote}
As we know, the challenge revolved around the meta-theory of
$\mathtt{F_{<:}}$, which, requiring induction over \emph{open} terms,
was an improvement over the gold standard of mechanized meta-theory in
the nineties: type soundness. Yet, the spotlight of the PC was still on
\begin{quote}
  ``type preservation and soundness theorems, unique decomposition
  properties of operational semantics, proofs of equivalence between
  algorithmic and declarative versions of type systems, etc.''
  (\emph{ibidem})
\end{quote}

Further, the authors made paramount ``the problem of representing and
reasoning about inductively-defined structure with \emph{binders}'' (our emphasis), while
providing a balanced criticism of de Bruijn indexes as an encoding
technique. That focus was understandable, since at that time the
only alternative to concrete representations was higher-order abstract
syntax (HOAS), mostly in the rather peculiar Twelf setting, the
implementation of nominal logic being in its infancy.

While the response of the theorem-proving community was impressive
with more than 15 (partial) solutions submitted
(\url{https://www.seas.upenn.edu/~plclub/poplmark/}), one can argue
whether the envisioned future has became our present --- according to
Sewell's POPL 2014 Program Chair's Report
(\url{https://www.cl.cam.ac.uk/~pes20/popl2014-pc-chair-report.pdf})
``Around 10\% of submissions were completely formalised, slightly more
partially formalised''. It is also debatable whether the challenge had
a direct impact on the development of proof assistants and logical
frameworks: specialized systems such as Abella~\citep{BaeldeCGMNTW14}
and Beluga~\citep{PientkaC15} were born out of independent research of
the early 2000. To be generous, we could impute Abella's generalization
of its specification logic to higher-order~\citep{hoabella} to this
Twelf POPLMark solution~\citep{Pientka07}, but development in
mainstream systems such as Coq, Agda, and (Nominal) Isabelle were
largely driven by other (internal) considerations.


In a much more modest setting, but in tune  with the goal of the PC,
\cite{FMP17} recently presented some benchmarks with the intention of going
beyond the issue of representing binders, whose pro and cons they
consider well-understood. Rather, the emphasis was on the all
important and often neglected issue of reasoning within a
\emph{context of assumptions}, and the role that properties such as
weakening, ordering, subsumption play in formal proofs. These
are more or less supported  in systems featuring some form of 
hypothetical and parametric reasoning, but the same issues occur in
first-order representation as well; in this setting, typically, they are not
recognized as crucial, rather they are considered one of the prices one
has to pay when reasoning over open terms. This set of benchmarks was
accompanied by a preliminary design of a common language and open
repository~\citep{FeltyMP15}, which is fair to say did not have a
resounding impact so far.
% Those benchmarks were by design hand-crafted in their simplicity to highlights

In the mean time, the PL world did not stand still, obviously.  One
element that we have picked on is the multiplication of the use of
proofs by \emph{logical relations}~\citep{Statman85} --- not
coincidentally, those featured in~\cite{Aydemir05TPHOLs}'s section ``Beyond the
challenge''.  From the go-to technique to prove normalization of
certain calculi, proofs by logical relations are now used to attack
problems in the theory of complex languages models, with 
applications to issues in equivalence of programs, {compiler
  correctness}, representation independence and even more intensional
properties such as non-interference, differential privacy and secure
multi-language inter-operability, to cite just a
few~\citep{Ahmed15,BowmanA15,NeisHKMDV15}.

Picking up on PC's final remark ``We will issue a small number of further
challenges [\dots]'', we propose, as we detail in
Section~\ref{ssec:expl} a new challenge that we hope it will move the
bar a bit forward. We suggest Strong Normalization (SN) for the
simply-typed lambda-calculus proven via logical relations in the
Kripke style formulation, see~\citep{Coquand91} for an early use. We
discuss the rationale in the next Section.

%%% Local Variables:
%%% mode: latex
%%% TeX-master: "poplr"
%%% End:

%  LocalWords:  TPTP POPLMark Aydemir TPHOLs POPL ICFP checkable de
%  LocalWords:  Bruijn HOAS Twelf TP Sewell's formalised Abella Coq
%  LocalWords:  BaeldeCGMNTW PientkaC Abella's hoabella Pientka Agda
%  LocalWords:  subsumption FeltyMP Statman intensional multi BowmanA
%  LocalWords:  operability NeisHKMDV Kripke Coquand Altenkirch tocl
%  LocalWords:  reducibility renamings girardLafontTaylor

\section{The Challenge}
\label{sec:chal}
% \begin{metanote}
%   Intentionally using same sections of POPLMark
% \end{metanote}

\subsection{Problem Selection}
\label{ssec:select}

(Strong) normalization by Tait's method is a well-understood and
reasonably circumscribed problem that has been a cornerstone of
mechanized PL theory, starting from~\citep{Altenkirch93}. There are of
course many alternative ways to prove SN for a lambda-calculus, see
for example the inductive approach of~\citep{Joachimski2003},
partially formalized in~\citep{ABEL20083}, or by reduction from strong
to weak normalization~\cite{SORENSEN199735}. For that matter, a SN
proof via logical relations for the simply-typed lambda calculus can
be carried out (see for a classic example~\citep{girardLafontTaylor})
{without} appealing to a Kripke definition of reducibility, at the
cost, though, of a rather cavalier approach to ``free''
variables. However, the Kripke technique is handy in establishing SN for richer
theories such as dependently typed ones, as well as for proving
stronger results, for example about equivalence
checking~\citep{Crary:ATAPL,Harper03tocl}.



We claim that mechanizing such a proof is
indeed challenging since:

\begin{itemize}
\item It focus on reasoning on \emph{open} terms and on relating
  different contexts or \emph{worlds}, taking seriously the Kripke
  analogy. The quantification over \emph{all} extensions of the given
  world may be problematic for frameworks where contexts are only
  implicitly represented, or, on the flip side, may require several
  boring weakening lemmas in first-order representations.
\item The definition of \emph{reducibility} requires a sophisticated
  notion of inductive definition, which must be compatible with the
  binding structures, but also be able to take into account
  \emph{stratification}, to tame the negative occurrence of the
  defined notion.
\item Simultaneous substitutions and their equational theory
  (composition, commutation etc.) are central in formulating and
  proving the main result. For example, in the proof of the Fundamental
  Theorem~\ref{thm:fund}, we need to push substitutions through
  (binding) constructs.
\end{itemize}
%
In this sense, this challenge goes well beyond the original PC, where
the emphasis was on binder representations, proofs by structural
induction and operational semantics animation.


\smallskip

Previous formalizations of strong normalization usually follows
Girard's approach, see for example~\cite{DonnellyX07} carried out in
ATS/LF, or the one available in the Abella repository
(\url{abella-prover.org/~normalization/}).  Less frequent are
formalizations following the Kripke discipline: both~\cite{CaveP15}
and~\cite{NarbouxU08} encode~\cite{Crary:ATAPL}'s account of decision
procedures for term equivalence in the STLC, in Beluga and Nominal
Isabelle respectively; the latter was then extended
in~\citep{Urban2011} to formalize the analogous result for
LF~\citep{Harper03tocl}. See~\citep{AbelV14} for a SN Kripke-style
proof for a more complex calculus and~\citep{Rabe:2013} for another
take to handling dependent types --- this paper also contains many
more references to the literature.


%% this bit is new

The choice of a Kripke-style proof of SN for the STLC may sound
contentious on several grounds and hence we will try to motivate it
further:

\begin{itemize}
\item We acknowledge that SN is not the most exciting application of logical
  relations, some of which we have mentioned in the previous Section. Still, it
  is an important topic in type theory, in particular w.r.t.\ logical
  frameworks' meta-theory, see for example~\citep{AltenkirchK16}, and
  in this sense dear to our hearts. It is  a well-known textbook example,
  which uses techniques that should be familiar to the community of
  interest in the simplest possible setting.
\item Yes, the STLC is the prototypical toy language, while a
  POPL paper will address richer PL theory
  aspects. For one, adding more constructs, say in the PCF direction,
  perhaps with an iterator, would make the proof of the fundamental
  theorem longer, but not more interesting. Secondly, we think that a
  good benchmark should be simple  enough that it
  could be tried out almost immediately if one is acquainted with
  proof-assistants. Conversely, it should encourage a PL theorist to
  start playing with proof assistants. Finally, we do suggest extensions
  of our challenge in the next Section.
\item The requirement of the ``Kripke-style'' may seem overly
  constrictive, especially since this may  not be strictly needed for
  the STLC\@. However, as we have argued before, this is meant as a
  springboard for more complex case studies, where this technique
  is forced on us. Remember that we are interested in \emph{comparing} solutions. A more
  ambitious challenge may not solicit enough solutions, if the problem
  is too exotic or simply too lengthy.
\end{itemize}



% Therefore we insist that the proof must be Kripke-style, others do not
% apply!
\subsection{Evaluation Criteria}
\label{ssec:ev}
One of the limitations of the PC experiment was in the
\emph{evaluation} of the solutions, although it is not easy to avoid
the ``trip to the zoo'' effect, well-known from trying to comparing programming
languages: there is no theory underlying the evaluation; criteria
tend to be rather qualitative, and finally, the comparison itself may
be lengthy~\citep{companion}. Within these limitations, of the
proposed solutions we will take into consideration the:
\begin{itemize}
\item  Size of the necessary infrastructure for defining the base language:
    binding, substitutions, renamings, contexts, together with
    substitution and other infrastructural lemmas.
  \item Size of the main development versus the main theorems in the
    on-paper proof, in particular, number of technical lemmas not
    having a direct counterpart in the on-paper proof.
\end{itemize}
More qualitatively, we will try to assess the:
\begin{itemize}
\item Ease of using the infrastructure for supporting binding,
  contexts, etc. How easy is it to apply the appropriate lemmas in the
  main proof? For example, does applying the equational theory of
  substitutions require low-level rewriting, or is it automatic?
\item Ease of development of the overall proof; what support is
  present for proof construction, when not for proof and counterexample
  search?
\end{itemize}
\subsection{The Challenge, Explained}
\label{ssec:expl}






Let us recall the definition of the STLC, starting with the grammar of terms, types, contexts and substitutions:
\[
\begin{array}{ll@{\bnfas}l}
\mbox{Terms} & M, N & x \bnfalt \lam x{:}T.M \bnfalt M \app N \\
\mbox{Types} & T, S & B \bnfalt T \arrow S\\
%%\mbox{Values} & V & \lam x.M 
\mbox{Context} & \Gamma & \cdot\bnfalt\Gamma, x\oftp T\\
\mbox{Subs} & \sigma & \epsilon\bnfalt\sigma, N/x
\end{array}
\]
The static and dynamic semantics are standard and are depicted in
Figure~\ref{fig:stlc}. Since we want to be very upfront about the fact
that evaluation goes under a lambda and thus involves open terms, we
make the context explicit even in the reduction rules, contrary to
what, say, Barendregt would do. Note that, because of rule \EAbsStep,
we do not need to assume that the base type is inhabited by a
constant.  We denote with $[\sigma] M$ the application of the
simultaneous substitution $\sigma$ to $M$ and with
$[\sigma_1]\sigma_2$ their composition.
% \unsure{going for church typing
% to have one form of ctx even in step -am}
\begin{figure*}[t!]
  \centering
  \[
\begin{array}{c}
\multicolumn{1}{l}{\fbox{$\Gamma \vdash \tmhastype M
    T$}\quad\mbox{Term $M$ has type $T$ in context $\Gamma$} }
\\[1em]
\infer[u]{\Gamma \vdash \tmhastype x T}{\tmhastype x T \in \Gamma} \qquad
\infer[\TFn^{x}]{\Gamma \vdash \tmhastype {(\lam x\oftp T.M)} {(T \arrow S)}}
                 {\Gamma, \tmhastype x T \vdash \tmhastype M S}
\qquad %\\[1em]
\infer[\TApp]{\Gamma \vdash \tmhastype {(M \app N)} S}
             {\Gamma \vdash \tmhastype M (T \arrow S)
  & \Gamma \vdash \tmhastype N T}\\[1em]
\end{array}
\]
%
\[
\begin{array}{c}
\multicolumn{1}{l}{\fbox{$\Gamma \vdash M \Steps M'$}\quad\mbox{Term $M$ steps to term $M'$ in  context $\Gamma$}}
\\[1em]
\infer[\EAbsStep^x]{\Gamma \vdash\lam x\oftp T.M \Steps \lam x\oftp T.M'}{\Gamma,x\oftp T \vdash M \Steps M'} \quad
\infer[\EAppBeta]{\Gamma \vdash (\lam x\oftp T.M) \app N \Steps [N/x]M}{} \quad %\\[1em]
\infer[\EAppArgStep]{\Gamma \vdash M \app N \Steps M'\;N}{\Gamma \vdash M \Steps M'} \quad
\infer[\EAppFnStep]{\Gamma \vdash M \app N \Steps M\;N'}{\Gamma \vdash N \Steps N'}
% \\[1em]
% \multicolumn{1}{l}{\fbox{$M \MSteps M'$}~~ \mbox{Term $M$ steps in
%     multiple steps to term $M'$}}\\[1em]
% \infer[\MRef]{M \MSteps M}{} \qquad
% \infer[\MOne]{M \MSteps M'}{M \Steps N & N \MSteps M'} 
\end{array}
\]

  \caption{Typing and reduction rules for the STLC}
  \label{fig:stlc}
\end{figure*}

We now define the set of \emph{strongly-normalizing} terms as
pioneered by~\cite{Altenkirch93} and by now usual:
\[
\infer[\mathit{SN-WF}]{\sn \Gamma M}{\forall M'.~\Gamma\vdash M\Steps M'\quad \sn \Gamma {M'}}
\]
expressing that the set of strongly normalizing terms is the
well-founded part of the reduction relation. A more explicit
formulation of strong normalization is allowed, see for
example~\citep{Joachimski2003}, but then an equivalence proof should
be provided. Note that reasoning with the above rule \textit{SN-WF}
cannot proceed by structural induction, since it is not the case that
$M'$ is a sub-term of $M$.
%

The logical predicates have the following structure:
\begin{itemize}
\item  $\rcs \Gamma M T$, and
\item $\rcs {\Gamma'} \sigma \Gamma$.
\end{itemize}


We use a Kripke-style logical relations definition where we
\emph{witness} the context extension using a \emph{weakening} substitution
$\rho$. This can be seen as a \emph{shift} in de Bruijn terminology, while 
%% something about shifting in DB, no need in named approaches 
% This treatment is inspired by Beluga (see,
% e.g.,~\citep{CaveP15}) to underline the context reasoning, for which
% $M$ depends on variable in $\Gamma$ and needs to be weakened to live
% in world $\Delta$.
other encodings may use different (or no particular) implementation
 techniques for handling context extensions.


\begin{definition}
 [Reducibility Candidates]
\mbox{}
\begin{itemize}
\item $\rcs \Gamma M B$ iff $\Gamma\vd M\hastype B$ and $\sn \Gamma M$:
\item $\rcs \Gamma M {T\arrow S}$ iff
  $ \Gamma\vd M\hastype {T\arrow S}$ and for all $N,\Delta$ such that
  $\Gamma \leq_\rho \Delta$, if $\rcs \Delta N {T}$ then
  $\rcs \Delta {([\rho]M) \app N} {S}$.
\end{itemize}
\end{definition}


As usual, we lift reducibility to substitutions:
\begin{definition}[Reducible Substitutions]\mbox{}
  \begin{itemize}
  \item \rcs {\Gamma'} {\epsilon} \cdot
\item $\rcs {\Gamma'} {\sigma, N/x}  {\Gamma, x:T} $ % \\ 
  iff 
$\rcs  {\Gamma'} \sigma \Gamma $ and  
$\rcs  {\Gamma'} N T$.
  \end{itemize}
\end{definition}

%We use $\rho$ for a renaming substitution. 
We now give an outline of the proof as a sequence of lemmas --- the
reader will find all the details in the forthcoming full version of this paper.

\begin{lemma}[Semantic Function Application]\mbox{}\\
If $\rcs {\Gamma} {M} {T \arrow S}$
and $\rcs \Gamma N T$
%\\
then $\rcs \Gamma {M~N} S$.
\end{lemma}
\begin{proof}
  Immediate, by definition.
\end{proof}

\begin{lemma}[SN Closure under Weakening]
\label{le:sn-clo}\mbox{} \\
If $\Gamma_1\leq_\rho \Gamma_2$ % and $\Gamma_1 \vdash M : A$
 and $\sn {\Gamma_1} M$ 
then $\sn {\Gamma_2} {[\rho]M}$. 
\end{lemma}
\begin{proof}
By induction on the derivation of $\sn {\Gamma_1} M$.  
\end{proof}

\begin{lemma}[Closure of Reducibility under Weakening]\mbox{} \\
If $\Gamma_1 \leq_\rho \Gamma_2$ and 
 $\rcs {\Gamma_1} M T$ then $\rcs {\Gamma_2} {[\rho]M} T$. 
\end{lemma}
\begin{proof}
  By cases on the definition of reducibility using the above
  Lemma~\ref{le:sn-clo} and weakening for typing.
\end{proof}

\begin{lemma}[Weakening of Reducible Substitutions]\label{lem:weakredsub}\mbox{} \\
If $\Gamma_1 \leq_\rho \Gamma_2$ and 
  $\rcs {\Gamma_1} \sigma \Phi$ 
then $\rcs {\Gamma_2} {[\rho]\sigma} \Phi$.
\end{lemma}
\begin{proof}
  By induction on the derivation of $\rcs {\Gamma_1} \sigma \Phi$
  using Closure of Reducibility under Weakening.
\end{proof}
% \[
%   \begin{array}{lcl}
% \mbox{Evaluation Context}~E  & \bnfas & [~~] \mid E~M
%   \end{array}
% \]

% \begin{lemma}[Closue under Weak Head Expansion]\mbox{}\\
% If $\sn \Gamma N$ and 
% $\rcs \Gamma {E[~[N/x]M~]} T$ \\
% then 
% $\rcs \Gamma {E[~(\lambda x{:}A.M)~N~]} T$  
% \end{lemma}


\begin{lemma}[Closure under Beta Expansion]\label{lem:betaclosed}\mbox{} \\
If $\sn \Gamma N$ 
and $\rcs \Gamma {[N/x]M} S$ %\\
then $\rcs \Gamma {(\lambda x{:}T.M)~N} S$.
\end{lemma}
\begin{proof}
  By induction on $S$ after a suitable generalization.
\end{proof}
%
\begin{theorem}[Fundamental Theorem]
\label{thm:fund}\mbox{} \\
If $\Gamma  \vdash M : T$ 
and $\rcs {\Gamma'} \sigma \Gamma$ 
% \\ 
then $\rcs {\Gamma'} {[\sigma]M} T$.
\end{theorem}
\begin{proof}
  By induction on $\Gamma \vdash M : T$. In the case for functions, we
  use Closure under Beta Expansion (Lemma~\ref{lem:betaclosed}) and
  Weakening of reducible substitution (Lemma~\ref{lem:weakredsub}).
\end{proof}



%%% Local Variables:
%%% mode: latex
%%% TeX-master: "poplr"
%%% End:

%  LocalWords:  TODO Kripke Girard's Matthes Gandy Girard Abella LF
%  LocalWords:  DonnellyX LICS tocl CaveP TAPL Moggi’s renamings STLC
%  LocalWords:  Barendregt iff Schurmann Tait's Altenkirch Joachimski
%  LocalWords:  SORENSEN girardLafontTaylor Crary ATAPL AbelV Rabe de
%  LocalWords:  POPL ICFP Bruijn reducibility AltenkirchK PCF WF


\section{Beyond the Challenge}
\label{sec:beyond}
There is an ongoing tension between weak and strong logical
frameworks~\citep{DeBruijn91lf}, with which we can encode our
benchmarks. Weak frameworks are designed to accommodate advanced
infra-structural features for binders (HOAS/nominal syntax etc.) and
for judgments (hypothetical and parametric), but may struggle on other
issues, such as facilities for computation or higher-order
quantification/impredicativity. There are at least two coordinates in
which we can directly extend our benchmark, to further highlight this
dilemma:
\begin{itemize}
\item Logical relations for dependent types~\citep{Rabe:2013,AbelV14},
  up to the Calculus of Constructions. Here we need to go beyond
  first-order quantification, which is typically what is on offer in
  weak frameworks.
\item Proof by logical relation via
  \emph{step-indexing}~\citep{Appel:2001}. Here we have two issues:
  \begin{enumerate}
  \item the logical relation may even be harder to be accepted by the
    meta-language as an inductive definition 
% in its original   formulation 
than with simple types; in fact, the work around the
    negative occurrence of the defined relation cannot be based on
    structural induction on types, but it has to use some form of
    course-of-value induction.

\item   It involves a limited amount of arithmetic reasoning:
  \begin{quote}
    ``definitions and proofs have a tendency to become
  cluttered with extra indices and even arithmetic, which are really
  playing the role of construction line.'' (\cite{BentonH10}).
  \end{quote}
  This latter point may be problematic for frameworks such as Abella and Beluga,
  which do not (yet) have extensive libraries, nor computational mechanisms
  (rewriting, reflection) for those  tasks.
  \end{enumerate}
\end{itemize}

\section{Call for action}

We ask the community to submit solutions and 
%  - Give a short (informal) talk at LFMTP to present their solution
we plan to invite everyone who does so to contribute
towards a joint paper discussing trade-offs between them. The authors
commit themselves to produce solutions in Agda,
Abella and Beluga. To resurrect the slogan from the PC, a small step
(excuse the pun) for us, a big step for bringing mechanized
meta-theory to the masses!


%%% Local Variables:
%%% mode: latex
%%% TeX-master: "poplr"
%%% End:

%  LocalWords:  DeBruijn lf Appel Coq Agda Abella HOAS Rabe
%  LocalWords:  impredicativity


\bibliographystyle{abbrvnat}
\bibliography{p}
% \appendix

% \section{Appendix: Sketch of the Proof}
% \label{sec:app}


\end{document}

%%% Local Variables:
%%% mode: latex
%%% TeX-master: t
%%% End:

%  LocalWords:  POPLMark Andreas Gothenburg Chalmers Momigliano degli
%  LocalWords:  Universit Studi di Milano Pientka McGill metatheory
%  LocalWords:  Kripke
