\usepackage{xspace}
\usepackage{listings}

% ---------------------------------------------------------------------------
% ------------------------------ Contextual ML ------------------------------
% ---------------------------------------------------------------------------

\lstdefinelanguage{ContextualML}
{
  morekeywords={and, block, case, of, mlam, fn, impossible, let, in, schema,
    some, rec, type, ctype, prop, stratified, inductive, coinductive, LF, if, then,
    else, total},
  keepspaces=true,
  sensitive,
  morecomment=[l]{\%},
  morecomment=[n]{\%\{}{\}\%},
  morestring=[b]"
}[keywords,comments,strings]

\lstloadlanguages{ContextualML}
\lstset{language=ContextualML}

\newdimen\zzlistingsize
\newdimen\zzlistingsizedefault
\zzlistingsizedefault=10pt
\zzlistingsize=\zzlistingsizedefault
\global\def\CommentCopter{0}
\newcommand{\Lstbasicstyle}{\fontsize{\zzlistingsize}{1.05\zzlistingsize}\ttfamily%
}
\newcommand{\keywordcopter}{\fontsize{0.95\zzlistingsize}{1.0\zzlistingsize}\bf}
\newcommand{\stupidcopter}{\if0\CommentCopter\keywordcopter\fi}
\newcommand{\commentcopter}{\def\CommentCopter{1}\fontsize{0.95\zzlistingsize}{1.0\zzlistingsize}\rmfamily\slshape}

\newcommand{\caret}{\char94}

\newcommand{\LST}{\setlistingsize{\zzlistingsizedefault}}

\newlength{\zzlstwidth}
\newcommand{\setlistingsize}[1]{\zzlistingsize=#1%
\settowidth{\zzlstwidth}{{\Lstbasicstyle~}}%
%\setlength{\zzlstwidth}{3pt}%
}
\setlistingsize{\zzlistingsizedefault}

% The order of the "literate" definitions is significant:
%   later definitions shadow earlier ones.  The \\Pi definition must come
%   *after* the \\ definition, or the first part of \\Pi --- that is, \\ --- will
%   be matched, and instead of $\Pi$ you'll get $\lambda Pi$.
%
\lstset{literate={->}{{$\rightarrow~$}}2 %
                 {=>}{{$\Rightarrow~$}}2 %
                 {|-}{{$\vdash\,$}}2 %
                 {..}{{$.\hspace{-0.025cm}.\hspace{-0.025cm}.$}}1 % is there any nicer way?
                 {\\}{{$\lambda$}}1 %
                 {\\Pi}{{$\Pi$}}1 %
                 {\\gamma}{{$\gamma$}}1 %
                 {\\psi}{{$\psi$}}1 %
                 {\\sigma}{{$\sigma$}}1 %
                 {FN}{{$\Lambda$}}1 %
                 {<<}{\color{ForestGreen}}1 %
                 {<<r}{\color{FireBrick}}1 %
                 {<*}{\color{ForestGreen}}1 %
                 {<dim}{\color{DimGrey}}1 %
                 {>>}{\color{black}}1 %
                 {?}{\bf{?}}1,
        columns=[l]fullflexible,
        basicstyle=\ttfamily\lst@ifdisplaystyle\footnotesize\fi,
        keywordstyle=\bf,
        identifierstyle=\relax,
        stringstyle=\relax,
        commentstyle=\slshape\color{DimGrey},
        breaklines=true,
        % breakatwhitespace=true,   % doesn't do anything (?!)
        mathescape=true,   % interprets $...$ in listing as math mode
        xleftmargin=0.5cm,
      }


% ---------------------------------------------------------------------------

% %                               {^}{{$\caret$}}1 %
% \lstset{literate=%
% %                               {=>}{{$\Rightarrow~$}}2 %
% %                               {==>}{{$\Longrightarrow~$}}3 %
%                                {\\/}{{$\unty$}}2 %
% %                               {-}{-}1 %
% %                               {->}{{$\rightarrow~$}}2 %
%                                {^}{{$\,\caret\,$}}1 %
%                                {|-}{{$\vdash$}}1 %
%                                {orelse}{{\stupidcopter orelse}$~$}3 %
%                                {as}{{\stupidcopter as}$~$}1 %
%                                {else}{{\stupidcopter else}$~$}3 %
%                                {case}{{\stupidcopter case}$~$}3 %
%                                {mlam}{{\stupidcopter mlam}$~$}3 %
%                                {raise}{{\stupidcopter raise}$~$}4 %
%                                {let}{{\stupidcopter let}$~$}2 %
%                                {datatype}{{\stupidcopter datatype}$~$}6 %
%                                {datasort}{{\stupidcopter datasort}$~$}6 %
%                                {@@@}{\text{\$}}1 %
%                ,
%                columns=[l]fullflexible,
%                basewidth=\zzlstwidth,
%                basicstyle=\Lstbasicstyle,
%                keywordstyle=\keywordcopter,
%                identifierstyle=\relax,
% %               stringstyle=\relax,
%                commentstyle=\commentcopter,
% %               prebreak={\mbox{$\space\swarrow$}},
% %               postbreak={\mbox{$\space\searrow$}},
%                showstringspaces=false,
%                breaklines=true,
%                breakatwhitespace=true,
%                mathescape=true,
% %               tabsize=8,
%                texcl=false}

% \setlistingsize{11pt}

% \usepackage{pstricks,pst-node,pst-tree}
\usepackage{graphics}
\usepackage{graphicx}

\newdimen\zzfontsz
\newcommand{\fontsz}[2]{\zzfontsz=#1%
{\fontsize{\zzfontsz}{1.2\zzfontsz}\selectfont{#2}}}

\newcommand{\mathsz}[2]{\text{\fontsz{#1}{$#2$}}}

\newtheorem{definition}{Definition}[section]
\newtheorem{theorem}{Theorem}[section]
\newtheorem{conjecture}[theorem]{Conjecture}
\newtheorem{corollary}[theorem]{Corollary}
\newtheorem{proposition}[theorem]{Proposition}
\newtheorem{lemma}[theorem]{Lemma}

\renewcommand{\Tilde}{\textsf{\char"7E}}

\newcommand{\arrayenv}[1]{\renewcommand{\arraystretch}{1} \begin{array}[t]{@{}c@{}}#1\end{array}}
\newcommand{\arrayenvc}[1]{\renewcommand{\arraystretch}{1} \begin{array}[c]{@{}c@{}}#1\end{array}}
\newcommand{\arrayenvr}[1]{\renewcommand{\arraystretch}{1} \begin{array}[t]{@{}r@{}}#1\end{array}}
\newcommand{\arrayenvbr}[1]{\renewcommand{\arraystretch}{1} \begin{array}[b]{@{}r@{}}#1\end{array}}
\newcommand{\arrayenvl}[1]{\renewcommand{\arraystretch}{1} \begin{array}[t]{@{}l@{}}#1\end{array}}
\newcommand{\arrayenvb}[1]{\renewcommand{\arraystretch}{1}  \begin{array}[b]{@{}c@{}}#1\end{array}} 
\newcommand{\arrayenvbl}[1]{\renewcommand{\arraystretch}{1}  \begin{array}[b]{@{}l@{}}#1\end{array}}

\newcommand{\subtype}{\leq}

\newcommand{\union}{\mathrel{\cup}}
\newcommand{\sect}{\mathrel{\cap}}

\newcommand{\unit}{\texttt{()}}
\newcommand{\Unit}{\textsf{unit}}
\newcommand{\bang}{\texttt{!}}
\renewcommand{\gets}{\mathop{\texttt{:=}}}

\newcommand{\down}{\mathrel{\,\Downarrow\,}}
\newcommand{\step}{\mathrel{\,\Rightarrow\,}}
\newcommand{\mstep}{\longrightarrow^*}

\newcommand{\D}{\mathcal{D}}
\newcommand{\E}{\mathcal{E}}
\newcommand{\F}{\mathcal{F}}

\newcommand{\Rsectintro}{\textsc{$\sectty$intro}\xspace}
\newcommand{\Rsectelim}[1]{\textsc{$\sectty$elim{#1}}\xspace}

\newcommand{\TIf}{\textsc{t-if}}
\newcommand{\TPlus}{\textsc{t-plus}}
\newcommand{\TMult}{\textsc{t-mult}}
\newcommand{\TEq}{\textsc{t-eq}}
\newcommand{\TApp}{\textsc{t-app}}
\newcommand{\TSub}{\textsc{t-sub}}
\newcommand{\TFn}{\textsc{t-fn}}
\newcommand{\TFun}{\textsc{t-fun}}
\newcommand{\TPair}{\textsc{t-pair}}
\newcommand{\TFst}{\textsc{t-fst}}
\newcommand{\TSnd}{\textsc{t-snd}}
\newcommand{\TVar}{\textsc{t-var}}
\newcommand{\TNum}{\textsc{t-num}}
\newcommand{\TTrue}{\textsc{t-true}}
\newcommand{\TFalse}{\textsc{t-false}}

\newcommand{\TBinaryPrimop}{\textsc{t-binary-primop}\xspace}
\newcommand{\TUnaryPrimop}{\textsc{t-unary-primop}\xspace}
\newcommand{\TTuple}{\textsc{t-tuple}\xspace}
\newcommand{\TTupleSyn}{\textsc{t-tuple-syn}\xspace}
\newcommand{\TRec}{\textsc{t-rec}\xspace}
\newcommand{\TAnno}{\textsc{t-anno}\xspace}

\newcommand{\TLet}{\textsc{t-let}\xspace}
\newcommand{\TLetSyn}{\textsc{t-let-syn}\xspace}
\newcommand{\TDecs}{\textsc{t-decs}\xspace}
\newcommand{\TByName}{\textsc{t-by-name}}
\newcommand{\TByVal}{\textsc{t-by-val}}
\newcommand{\TByValTuple}{\textsc{t-by-val-tuple}}

\newcommand{\SIFT}{\textsc{s-iftrue}\xspace}
\newcommand{\SIFF}{\textsc{s-iffalse}\xspace}
\newcommand{\SIF}{\textsc{s-if}\xspace}
\newcommand{\SAppFnStep}{\textsc{s-app-fn}\xspace}
\newcommand{\SAppArgStep}{\textsc{s-app-arg}\xspace}
\newcommand{\SAppBeta}{\textsc{s-app}\xspace}

\newcommand{\BAnno}{\textsc{b-anno}\xspace}
\newcommand{\BAnnoFn}{\textsc{b-anno-fn}\xspace}
\newcommand{\BAnnoNonFn}{\textsc{b-anno-non-fn}\xspace}
\newcommand{\BIFT}{\textsc{b-iftrue}\xspace}
\newcommand{\BIFF}{\textsc{b-iffalse}\xspace}
\newcommand{\BOp}{\textsc{b-op}\xspace}
\newcommand{\BPlus}{\textsc{b-plus}\xspace}
\newcommand{\BEq}{\textsc{b-eq}\xspace}
\newcommand{\BLet}{\textsc{b-let}\xspace}
\newcommand{\BNum}{\textsc{b-num}\xspace}
\newcommand{\BVar}{\textsc{b-var}\xspace}
\newcommand{\BIF}{\textsc{b-if}\xspace}
\newcommand{\BTrue}{\textsc{b-true}\xspace}
\newcommand{\BFalse}{\textsc{b-false}\xspace}
\newcommand{\BFun}{\textsc{b-fun}\xspace}
\newcommand{\BRec}{\textsc{b-rec}\xspace}

\newcommand{\base}{\textsf{i}}
\newcommand{\Int}{\textsf{int}}
\newcommand{\Float}{\textsf{float}}
\newcommand{\Bool}{\textsf{bool}}
\newcommand{\Real}{\textsf{real}}
\newcommand{\String}{\textsf{string}}
\newcommand{\Char}{\textsf{char}}
\newcommand{\Ref}{~\textsf{ref}}
\newcommand{\Array}{~\textsf{array}}
\newcommand{\norm}{\mathrel{\,\uparrow\,}}
\newcommand{\neut}{\mathrel{\,\downarrow\,}}
\newcommand{\neutG}{\Gamma^{\downarrow}}
\newcommand{\syn}{\mathrel{\,\Rightarrow\,}}
\newcommand{\chk}{\mathrel{\,\Leftarrow\,}}
\newcommand{\arr}{\mathrel{\texttt{->}}}
% \newcommand{\arrow}{\arr}
\newcommand{\entails}{\vdash}
\newcommand{\such}{~|~}
\newcommand{\sectty}{\mathrel{\text{\&}}}

\newcommand{\tmtrue}{\textsf{true}}
\newcommand{\tmfalse}{\textsf{false}}
\newcommand{\tmif}[3]{\textsf{if\;} #1 \textsf{\;then\;} #2 \textsf{\;else\;} #3}
\newcommand{\tmfun}[3]{\textsf{fun } #1 (#2) = #3}
\newcommand{\tmfn}[2]{\textsf{fn } #1\;\texttt{=>}\;#2}
\newcommand{\tmapp}[2]{#1\;#2}
\newcommand{\tmrec}[3]{\textsf{rec } {#1}\,:\,{#2}\;\texttt{=>}\;#3}
\newcommand{\tmlet}[3]{\textsf{let } #1 = #2 \textsf{\;in\;} #3\; \textsf{end}}

 \newcommand{\tmfst}[1]{\textsf{fst}\;{#1}\xspace}
 \newcommand{\tmsnd}[1]{\textsf{snd}\;{#1}\xspace}

% Numerical expressions
\newcommand{\tmzero}{\textsf{z}}
\newcommand{\tmsucc}{\textsf{succ}}
\newcommand{\tmpred}{\textsf{pred}}
\newcommand{\tmiszero}{\textsf{iszero}}


\newcommand{\BLetn}{\textsc{b-letn}\xspace}
\newcommand{\BLetp}{\textsc{b-letpair}\xspace}
\newcommand{\BPair}{\textsc{b-pair}\xspace}
\newcommand{\BFst}{\textsc{b-fst}\xspace}
\newcommand{\BSnd}{\textsc{b-snd}\xspace}
\newcommand{\BFn}{\textsc{b-fn}\xspace}
\newcommand{\BApp}{\textsc{b-app}\xspace}

\newcommand{\unif}{\doteq}
\newcommand{\totp}{\Rightarrow}
\newcommand{\emp}{\emptyset}
\newcommand{\TT}{\textsf{tt}}

% \newcommand{\D}{{\cal{D}}}
\newcommand{\FV}{\mathsf{FV}}

% \newcommand{\m}[1]{\mbox{\lstinline!#1!}}
\newcommand{\mlam}[2]{\m{mlam}\; #1 \Rightarrow #2}
\newcommand{\fn}[2]{\m{fn}\;#1 \Rightarrow #2}
\newcommand{\rec}[2]{\m{rec}\;#1 = #2}
\newcommand{\boxm}[2]{\m{box}(#1.\,#2)}
\newcommand{\letboxm}[3]{\m{let}\,[#1]\;{=}\;#2\;\m{in}\;#3}
\newcommand{\casev}[2]{\m{case}\;#1\;\m{of}\;#2}
\newcommand{\casebox}[2]{\casev{#1}{#2}}
\newcommand{\casearm}[2]{#1 \Rightarrow #2}
\newcommand{\branch}[4] {\Pibox #1.#2:#3 \Rightarrow #4}

