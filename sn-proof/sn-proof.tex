\documentclass{article}
\usepackage{lmodern}%
\usepackage{enumitem}
\newtheorem{exercise}{Exercise}[section]
% \newenvironment{exercise}{\begin{@exercise}\rm}{\end{@exercise}}
\newenvironment{ADDITIONAL}[1]{#1}{}
\newenvironment{SOLUTION}[1]{\paragraph*{Solution}\begin{it}#1}{\end{it}}
     
%\oddsidemargin 0pt
%\evensidemargin \oddsidemargin
%\marginparwidth 0.5in
     

     
% \setlength{\topmargin}{15mm}    
\setlength{\topmargin}{5mm}    

%\setlength{\textwidth}{155mm}    
\setlength{\textwidth}{155mm}    
%\setlength{\textwidth}{165mm}    
\setlength{\textheight}{200mm}
%\setlength{\textheight}{195mm}

\setlength{\evensidemargin}{5mm}
\setlength{\oddsidemargin}{5mm}

\newcommand{\hs}[1]{\hspace{#1}}

% \usepackage{fullpage}
\usepackage{amsmath}
\usepackage{amsthm}
\usepackage{latexsym}
\usepackage{amsfonts}
\usepackage{listings}
\usepackage{srcltx}
\usepackage{charter}
\usepackage{euler}

\usepackage{latexsym}
\usepackage{amsthm}
\usepackage{amssymb}
\usepackage{amsfonts}
\usepackage{comment}

\ifdefined\studentversion
  \excludecomment{exercise}
  \excludecomment{solution}
  \excludecomment{additional}
\else
% \excludecomment{additional}
\includecomment{exercise}
\includecomment{solution}
\includecomment{additional}
\fi

\newtheorem{@problem}{Exercise}[section]
\newenvironment{problem}{\begin{@problem}\rm}{\end{@problem}}

\newtheorem{@sol}{Solution}[section]
\newenvironment{sol}{\begin{@sol}\rm}{\end{@sol}}

\newcommand{\ext}[1]{\geq_{#1}}

\usepackage{proof}
\usepackage{cdsty}

\newcommand{\nl}{\overline{n}}

\newtheorem{@axiom}{Axiom}
\newenvironment{axiom}{\begin{@axiom}\rm}{\end{@axiom}}
% \newtheorem{@theorem}{Theorem}[section]
% \newenvironment{theorem}{\begin{@theorem}\rm}{\end{@theorem}}

% \newtheorem{@lemma}{Lemma}[section]
% \newenvironment{lemma}{\begin{@lemma}\rm}{\end{@lemma}}



\input prelude

\usepackage{graphics}
\usepackage{graphicx}

\usepackage{lstextract}

% \includeonly{intro,assign,unif,memo,index,memo-implement,proving}
% \includeonly{index} 

\newcommand{\B}{\mathcal{B}}
\newcommand{\C}{\mathcal{C}}
\newcommand{\G}{\mathcal{G}}
\renewcommand{\H}{\mathcal{H}}
\renewcommand{\P}{\mathcal{P}}
\newcommand{\Q}{\mathcal{Q}}

\newcommand{\SN}{\mathsf{SN}}
\newcommand{\SNe}{\mathsf{SNe}}
\newcommand{\csn}{\mathsf{sn}}
\newcommand{\CR}{\textsf{CR}}
\newcommand{\red}{\longrightarrow}
\newcommand{\redsn}{\longrightarrow_\csn}
\newcommand{\redSN}{\longrightarrow_\SN}
\newcommand{\clos}[1]{\overline{#1}}
\newcommand{\imply}{\Longrightarrow}
\renewcommand{\vec}[1]{\overrightarrow #1}

\newcommand{\id}{\textsf{wk}}
\newcommand{\wk}{\textsf{wk}}

\def\lv{\mathopen{{[\kern-0.14em[}}}    % opening [[ value delimiter
\def\rv{\mathclose{{]\kern-0.14em]}}}   % closing ]] value delimiter
\newcommand{\A}{\mathcal{A}}
% \newcommand{\G}{\mathcal{G}}
\newcommand{\den}[1]{\lv #1 \rv}
\newcommand{\Den}[3][]{\den{#2}^{#1}_{#3}}
\newcommand{\dent}[2]{\llparenthesis#1\rrparenthesis_{#2}}


\long\def\ednote#1{\footnote{[{\it #1\/}]}\message{ednote!}}
% \long\def\note#1{\begin{quote}[{\it #1\/}]\end{quote}\message{note!}}

\begin{document}
We discuss here an alternative proof method for proving normalization. We will focus here on a \emph{semantic} proof method using \emph{saturated sets}. This proof method goes back to Girard (1972) building on some previous ideas by Tait.

The key question is how to prove that give a lambda-term, its evaluation terminates, i.e. normalizes.  Recall the lambda-calculus together with its reduction rules.


\[
\begin{array}{llcl}
\mbox{Terms}  & M,N & \bnfas & x \mid \lambda x{:A}. M \mid M\;N \\
\mbox{Types} & A, B & \bnfas & \base \mid A \arrow B
\end{array}
\]

We consider as the main rule for reduction (or evaluation) applying a term to an abstraction, called \emph{$\beta$-reduction}.

\[
\begin{array}{lcl}
%\multicolumn{3}{l}{\mbox{$\beta$-reduction}}  \\
\Gamma \vdash (\lambda x{:}A.M)\;N & \red & \Gamma \vdash [N/x]M \qquad\qquad\mbox{$\beta$-reduction}
\end{array}
\]

The $\beta$-reduction rule only applies once we have found a redex. However, we also need congruence rules to allow evaluation of arbitrary subterms. 

\[
\begin{array}{l}
\infer{\Gamma \vdash M\,N \red M'\,N}{\Gamma \vdash M \red M'} \qquad  
\infer{\Gamma \vdash M\,N \red M\,N'}{N \red N'}
 \qquad
\infer{\Gamma \vdash \lambda x{:}A.M \red \lambda x{:}A.M'}{\Gamma, x{:}A \vdash M \red M'}
\end{array}
\]

To allow for weak head reductions, we rely on evaluation contexts.

\[
\begin{array}{lcl}
\mbox{Evaluation Context}~C & \bnfas & {\_} ~\mid C~M  
\end{array}
\]


The question then is, how do we know that reducing a well-typed lambda-term will halt? - This is equivalent to asking does a well-typed lambda-term normalize, i.e. after some reduction steps we will end up in a normal form where there are no further reductions possible. Since a normal lambda-term characterizes normal proofs, normalizing a lambda-term corresponds to normalizing proofs and demonstrates that every proof in the natural deduction system indeed has a normal proof.

Proving that reduction must terminate is not a simple syntactic argument based on terms, since the $\beta$-reduction rule may yield a term which is bigger than the term we started with. We hence need to find a different inductive argument. For the simply-typed lambda-calculus, we can prove that while the expression itself does not get smaller,  the type of an expression is. This is a syntactic argument; it however does not scale to polymorphic lambda-calculus. We will here instead discuss a \emph{semantic} proof method where we define the meaning of well-typed terms using the abstract notion of \emph{reducibility candidates}. 


% Unlike all the previous proofs which were syntactic and direct based on the structure of the derivation or terms, semantic proofs 

\section{Semantic Interpretation}
Working with well-typed terms means we need to be more careful to
consider a term within its typing context. In particular, when we
define the semantic interpretation of $\Gamma \models M : A \arrow B$
we must consider all extensions of $\Gamma$ (described by $\Gamma'
\ext \rho \Gamma$) in which we may use $M$.

\begin{itemize}
\item $\Gamma \models M : \base$ iff $\Gamma \vdash M\hastype \base$ and $M$ is strongly normalizing, i.e. $(\Gamma \vdash M) \in \SN$.
\item $\Gamma \models M : A \arrow B$ iff for all $\Gamma' \ext{\rho} \Gamma$ and $\Gamma' \vdash N :A$ if $\Gamma' \models N : A \imply \Gamma' \models M[\rho]~N : B$,
\end{itemize}


% Weakening holds for the semantic interpretations.

% \begin{lemma}[Semantic Weakening]\ref{lm:sweak}
% If $\Gamma \models M : A$ then $\Gamma, x{:}C \models M : A$.  
% \end{lemma}

% We sometimes write these definitions more compactly as follows

% \[
% \begin{array}{llcl}
% \mbox{Semantic base type} & \den{o} & := & \SN  \\
% \mbox{Semantic function type} & \den{A \arrow B} & := & \{ M | \forall \Gamma' \ext{\rho} \Gamma,~\forall \Gamma' \vdash N : A.~ \Gamma'\ models N : A \ \in \den{A}. M\;N \in \den{B} \}
% \end{array}
% \]


\section{General idea}

We prove that if a term is well-typed, then it is strongly normalizing in  two steps:

\begin{description}
\item[Step 1] If $\Gamma \models M : A$ then $(\Gamma \vdash M) \in \SN$. 
\item[Step 2] If $\Gamma \vdash M : A$ and $\Gamma' \models \sigma : \Gamma$ then $\Gamma' \models [\sigma]M : A$.
\end{description}

Therefore, we can conclude that if a term $M$ has type $A$ then $M \in \SN$, i.e. $M$ is strongly normalizing and its reduction is finite, choosing $\sigma$ to be the identity substitution. 
% \\[1em]
% We remark first, that all variables are in the semantic type $A$ and variables are strongly normalizing, i.e. they are already in normal form.

% % \begin{lemma}~\\
%   \begin{itemize}
%   \item If $\Gamma \vdash x : A$ then $\Gamma \models x : A$
%   \item If $\Gamma \vdash x : A$ then $(\Gamma \vdash x) \in \SN$.
%   \end{itemize}
  
% \end{lemma}

% These are of course statements we need to prove.

\section{Defining strongly normalizing terms}
\subsection{Definition of strong normalization via accessibility relation}
Intuitively, a term $M$ is strongly normalizing, if there exists no infinite reduction sequence. Constructively, we can define strong normalization as follows:

\begin{definition}\label{def:norm}
A term $M$ is strongly normalizing, if all its reducts are strongly
normalizing.\\
\[
\infer{(\Gamma \vdash M) \in \csn}{\forall M'.~\Gamma \vdash M \red M' \imply (\Gamma \vdash M') \in \csn}
\]
\end{definition}




\begin{lemma}[Properties of strongly normalizing terms]\label{lem:psn}$\;$
  \begin{enumerate}
  \item\label{pp2} If $\Gamma \vdash M \in \csn$ and $\Gamma \vdash N \in \csn$
    then $\Gamma \vdash M~N \in \csn$.
  \item\label{pp3} If $\Gamma \vdash [N/x]M \in \csn$ and $\Gamma \vdash N \in
    \csn$ then $\Gamma, x{:}A \vdash M \in \csn$.
  \item\label{pp4} If $\Gamma, x{:}A \vdash M \in \csn$ then $\Gamma \vdash  \lambda x.M \in \csn$
  \item\label{pp5} If $\Gamma, x{:}A \vdash M \in \csn$ and $\Gamma \vdash N \in
   \csn$ then $\Gamma \vdash [N/x]M \in \csn$.  
  \item\label{pp6} If $\Gamma \vdash M~N \in \csn$ then $\Gamma \vdash M \in \csn$ and $\Gamma \vdash N \in \csn$.
  \item\label{pp7} If $\Gamma \vdash \lambda x{:}A.M \in \csn$ then $\Gamma, x{:}A \vdash M \in \csn$.

%   \item\label{pp2} If $\Gamma \vdash M \in \csn$
%                  and $\Gamma \vdash N \in \csn$ where $M$ is not a $\lambda$
%                 then $\Gamma \vdash M\;N \in \csn$. (we also have $\Gamma
%                 \vdash M\;N \red M'\;N$ and $\Gamma \vdash M'\;N \in \csn$ as i.h.)
  \end{enumerate} 
\end{lemma}
\begin{proof}
Properties \ref{pp6}, \ref{pp7}, and \ref{pp4} are proven by induction on the first derivation. We prove the Properties \ref{pp2}, \ref{pp3}, and \ref{pp5} mutually. 
\\[1em]
\fbox{\ref{pp2}. If $\Gamma \vdash M \in \csn$ and $\Gamma \vdash N \in \csn$
    then $\Gamma \vdash M~N \in \csn$. }
\\[1em]
By simultaneous induction on $\Gamma \vdash M \in \csn$ and $\Gamma \vdash N \in \csn$.\\[1em]
Assume $\Gamma \vdash M~N \red Q$
\paragraph{Sub-case:} $\D = \ianc
 {\Gamma \vdash M \red M'}
 {\Gamma \vdash M\,N \red M'\,N}{}$ and $Q = M'~N$
\\[1em]
$\Gamma \vdash M' \in \csn$ \hfill by using assumption $\Gamma \vdash M \in \csn$ \\
$\Gamma \vdash M'~N \in \csn$ \hfill by IH \\
$\Gamma \vdash Q \in \csn$ \hfill since $Q = M'~N$
% $\Gamma \vdash M~N \in \csn$ \hfill by abstraction over assumption $\Gamma \vdash M~N \red Q$.

\paragraph{Sub-case:} $\D = \ianc
 {\Gamma \vdash N \red N'}
 {\Gamma \vdash M\,N \red M\,N'}{}$ and $Q = M~N'$
\\[1em]
$\Gamma \vdash N' \in \csn$ \hfill by using assumption $\Gamma \vdash N \in \csn$ \\
$\Gamma \vdash M~N' \in \csn$ \hfill by IH \\
$\Gamma \vdash Q \in \csn$ \hfill since $Q = M~N'$
% $\Gamma \vdash M~N \in \csn$ \hfill by abstraction over assumption $\Gamma \vdash M~N \red Q$.

\paragraph{Sub-case:} $\D = \Gamma \vdash (\lambda x{:}A.M)\;N  \red  \Gamma \vdash [N/x]M$
\\[1em]
$\Gamma \vdash \lambda x{:}A.M \in \csn$ \hfill by assumption \\
$\Gamma \vdash N \in \csn$ \hfill by assumption \\
$\Gamma, x{:}A \vdash M \in \csn$ \hfill Property \ref{pp7}\\
$\Gamma \vdash [N/x]M \in \csn$ \hfill by Property \ref{pp5}
\\[1em]
$\Gamma \vdash M~N \in \csn$ \hfill since $\Gamma \vdash M~N \red Q$ was arbitrary.
\\[1em]
%
\fbox{\ref{pp3}. If $\Gamma \vdash [N/x]M \in \csn$ and $\Gamma \vdash N \in
    \csn$ then $\Gamma, x{:}A \vdash M \in \csn$.}
\\[1em]
By induction on $M$.
\paragraph{Sub-case:} $M = x$ \\
$\Gamma, x{:}A \vdash x \in \csn$ \hfill trivial since there is no reduction $x \red x$

\paragraph{Sub-case:} $M = M_1~M_2$ \\
$[N/x](M_1~M_2) = [N/x]M_1~[N/x]M_2$ \hfill by def. of subst. \\
$\Gamma \vdash [N/x]M_1 \in \csn$ \hfill by Property \ref{pp6} \\
$\Gamma \vdash [N/x]M_2 \in \csn$ \hfill by Property \ref{pp6} \\
$\Gamma, x{:}A \vdash M_1 \in \csn$ \hfill by IH \\
$\Gamma, x{:}A \vdash M_2 \in \csn$ \hfill by IH \\
$\Gamma, x{:}A \vdash M_1~M_2 \in \csn$ \hfill by Property \ref{pp2}
%

\paragraph{Sub-case:} $M = \lambda y{:}B.M'$ \\
$[N/x](\lambda y{:}B. [N/x]M'$ \hfill by def. subst. \\
$\Gamma, y{:}B \vdash [N/x]M' \in \csn$ \hfill by Property \ref{pp7}\\
$\Gamma, y{:}B, x{:}A \vdash M' \in \csn$ \hfill by IH \\
$\Gamma, x{:}A, y{:}B \vdash M' \in \csn$ \hfill by exchange \\
$\Gamma, x{:}A \vdash \lambda y{:}B.M' \in \csn$ \hfill by Property \ref{pp4}
\\[1em]
%
\fbox{\ref{pp4}. If $\Gamma, x{:}A \vdash M \in \csn$ then $\Gamma \vdash  \lambda x{:}A.M \in \csn$}
\\[1em]
Induction on $\Gamma, x{:}A \vdash M \in \csn$\\[1em]
Assume $\Gamma \vdash \lambda x{:}A.M \red Q$ \\
$\Gamma, x{:}A \vdash M \red M'$ and $Q = \lambda x{:}A.M'$ \hfill by reduction rule for $\lambda$.\\
$\Gamma \vdash M' \in \csn$ \hfill by assumption $\Gamma, x{:}A \vdash M \in \csn$ \\
$\Gamma \vdash \lambda x{:}A.M' \in \csn$ \hfill by IH \\
$\Gamma \vdash Q \in \csn$ \hfill since $Q = \lambda x{:}A.M'$\\
$\Gamma \vdash \lambda x.M \in \csn$ \hfill since $\Gamma \vdash \lambda x.M \red Q$ was arbitrary
\\[1em]
\fbox{\ref{pp5}. If $\Gamma, x{:}A \vdash M \in \csn$ and $\Gamma \vdash N \in
   \csn$ then $\Gamma \vdash [N/x]M \in \csn$.  }
\\[1em]
Induction on $M$.
% \\[1em]
\paragraph{Sub-case:} $M = x$
\\
$\Gamma \vdash [N/x]x \in \csn$ \hfill by assumption $\Gamma \vdash N \in \csn$
%
\paragraph{Sub-case:} $M = M_1~M_2$
\\
$\Gamma,x{:}A \vdash M_1~M_2 \in \csn$ \hfill by assumption \\
$\Gamma, x{:}A \vdash M_1 \in \csn$ and $\Gamma, x{:}A \vdash M_2 \in \csn$ \hfill by Property \ref{pp6} \\
$\Gamma \vdash [N/x]M_1 \in \csn$ \hfill by IH \\
$\Gamma \vdash [N/x]M_2 \in \csn$ \hfill by IH \\
$\Gamma \vdash [N/x]M_1~[N/x]M_2 \in \csn$ \hfill by Property \ref{pp2}\\
$\Gamma \vdash [N/x](M_1~M_2) \in \csn$ \hfill by subst. def.
%
\paragraph{Sub-case:} $M = \lambda y{:}B.M'$
\\
$\Gamma, x{:}A \vdash \lambda y{:}B.M' \in \csn$ \hfill by assumption \\
$\Gamma, x{:}A, y{:}B \vdash M' \in \csn$ \hfill by Property \ref{pp7}\\
$\Gamma, y{:}B, x{:}A \vdash M' \in \csn$ \hfill by exchange \\
$\Gamma, y{:}B \vdash N \in \csn$ \hfill by weakening\\
$\Gamma, y{:}B \vdash [N/x]M' \in \csn$ \hfill by IH \\
$\Gamma \vdash \lambda y{:}B. [N/x]M' \in \csn$ \hfill by Property \ref{pp4}\\
$\Gamma \vdash [N/x](\lambda y{:}B.M') \in \csn$ \hfill by subst. def
\\[1em]
\fbox{\ref{pp6}. If $\Gamma \vdash M~N \in \csn$ then $\Gamma \vdash M \in \csn$ and $\Gamma \vdash N \in \csn$.}
\\[1em]
We prove first: If $\Gamma \vdash M~N \in \csn$ then $\Gamma \vdash M \in \csn$. Proving $\Gamma \vdash M~N \in \csn$ implies also $\Gamma \vdash N \in \csn$ is similar.
\\[0.5em]
Assume $\Gamma \vdash M \red M'$\\
$\Gamma \vdash M~N \red M'~N$ \hfill by reduction rule for application \\
$\Gamma \vdash M'~N \in \csn$ \hfill by assumption $\Gamma \vdash M~N \in \csn$\\
$\Gamma \vdash M' \in \csn$ \hfill by IH\\
$\Gamma \vdash M \in \csn$ \hfill since $\Gamma \vdash M \red M'$ was arbitrary
\\[1em]
\fbox{\ref{pp7}. If $\Gamma \vdash \lambda x{:}A.M \in \csn$ 
    then $\Gamma, x{:}A \vdash M \in \csn$. }
\\[1em]
By induction on $\Gamma \vdash \lambda x{:}A.M \in \csn$.
\\[1em]
Assume  $\Gamma, x{:}A \vdash M \red M'$\\
$\Gamma \vdash \lambda x{:}A.M \red \lambda x{:}A.M'$ \hfill by reduction rules for $\lambda$\\
$\Gamma \vdash \lambda x{:}A. M' \in \csn$ \hfill by assumption $\Gamma \vdash \lambda x{:}A.M\in\csn$\\
$\Gamma, x{:}A \vdash M' \in \csn$ \hfill by IH\\
$\Gamma, x{:}A \vdash M \in \csn$ \hfill since  $\Gamma, x{:}A \vdash M \red M'$ was arbitrary
\\[1em]
\end{proof}


\begin{corollary}\label{cor:psn}~
  \begin{enumerate}
  \item \label{pp1} If $\Gamma \vdash   [N/x]M \in \csn$
                  and $\Gamma \vdash N \in \csn$ 
                 then $\Gamma \vdash (\lambda x.M) \;N \in \csn$.
  \end{enumerate}
\end{corollary}
\begin{proof}
  Property \ref{pp1} follows directly from Lemma
  \ref{lem:psn}(\ref{pp3}) and Lemma
  \ref{lem:psn}(\ref{pp4}).
\end{proof}




\begin{lemma}[Evaluation Contexts]\label{lm:ecxt}$\;$
If $\Gamma \vdash C[x]~N \red R$ then there exists an
    evaluation context $C'$ s.t. $R = C'[x]$.
\end{lemma}
\begin{proof}
By induction on the structure of evaluation contexts.

\paragraph{Case}  $C[x] = x$ 
\\[1em]
$\Gamma \vdash x~N \red R$ \hfill by assumption\\
Impossible since $x~N$ does not step.

\paragraph{Case} $C[x] = C_0[x]~M$ \\[1em]
$\Gamma \vdash (C_0[x]~M)~N \red R$ \hfill by assumption\\[1em]
\textbf{Sub-case} $\ianc{\Gamma \vdash (C_0[x]~M) \red R}{\Gamma  \vdash (C_0[x]~M)\,N \red R\,N}{}$
\\[1em]
There exists an evaluation context $C_1[x] = R$ \hfill by IH \\
Therefore there exists an evaluation context $C'[x] = C_1[x]~N$
\\[1em]
\textbf{Sub-case} $\ianc{\Gamma \vdash N \red N'}{\Gamma \vdash (C_0[x]~M)\,N \red (C_0[x]~M)\,N'}{}$
\\[1em]
Hence there exists an evaluation context $C'$ s.t. $C'[x] = (C_0[x]~M)\,N'$. 

\end{proof}

% Closure of SN under WH Expansion
% 1.
% If $\Gamma \vdash N \in \csn$
% and $\Gamma \vdash C[M[N/x]] \in \csn$ 
% then $\Gamma \vdash C[ (lam x.M)~N] \in \csn$.
% 2.
% If $\Gamma \vdash C[ (lam x.M)~N ] \red R$
% then $\Gamma \vdash C[ M[N/x] } \red R$
% 
% 3.

\begin{lemma}[Closure properties of strongly normalizing terms]\label{lm:closn}$\;$
  \begin{enumerate}
%   \item\label{cp1} If $\Gamma \vdash [N/x]M \in \csn$ and $\Gamma
%   \vdash N \in \csn$ then $\Gamma
%    \vdash M \in \csn$.
  \item\label{cp2} For all variables $x:A \in \Gamma$, $\Gamma \vdash x \in \csn$.
  \item\label{cp3} If $\Gamma \vdash C[x] \in \csn$ and $\Gamma \vdash N \in \csn$ 
     then $\Gamma \vdash C[x]\,N \in \csn$.
  \item\label{cp3b} If $\Gamma \vdash C[x]~M \red R$ and $\Gamma \in C[x] \in
\csn$ and $\Gamma \vdash M \in \csn$ then $\Gamma \vdash R \in \csn$. 
   \item\label{cp4} If $\Gamma,x{:}A \vdash M \in \csn$ then $\Gamma \vdash \lambda x.M \in \csn$.
   \item\label{cp5} {$\csn$-backward closure:} If $\Gamma \vdash M
     \redsn M'$ and $\Gamma \vdash M' \in \csn$ then
     $\Gamma \vdash M \in \csn$ where
\[
\begin{array}{l}
\infer{\Gamma \vdash (\lambda x.M)\;N \redsn [N/x]M}{\Gamma \vdash N \in \csn} 
\qquad
\infer{\Gamma \vdash M\;N \redsn M'\;N}{\Gamma \vdash M \redsn M'} %  M\,\text{is not a}\;\lambda %  %& s \in \csn
  \\[1em]
\end{array}
\]
\end{enumerate}
\end{lemma}
\begin{proof}\mbox{}~\\[1em]
\noindent

\fbox{\ref{cp2}.  For all variables $x:A \in \Gamma$, $\Gamma \vdash x \in \csn$.}
\\[1em]
$\forall M'.~\Gamma \vdash x \red M' \imply (\Gamma \vdash M') \in \csn$ \hfill since $\Gamma \vdash x \red M'$ is impossible
\\
$\Gamma \vdash x \in \csn$
\\[1em]
\fbox{
%  \begin{tabular}{lp{11cm}}
\ref{cp3}.  If $\Gamma \vdash C[x] \in \csn$ and $\Gamma \vdash N \in \csn$ 
     then $\Gamma \vdash C[x]\,N \in \csn$.\\
% \ref{cp3b}. & If $\Gamma \vdash C[x]~M \red R$ and $\Gamma \in C[x] \in
% \csn$ and $\Gamma \vdash M \in \csn$ then $\Gamma \vdash R \in \csn$.    
%  \end{tabular}
 }

\paragraph{Case}$\D = 
\ianc{\forall M'.~\Gamma \vdash C[x] \red M' \imply (\Gamma \vdash M') \in \csn}
     {(\Gamma \vdash C[x]) \in \csn}{}$
\\[1em]
\textbf{Sub-case}: $C[x] = x$ 
\\
impossible
\\[1em]
\textbf{Sub-case}: $C[x] = C_0[x]~M$ 
\\[1em]
Assume $\Gamma \vdash (C_0[x]~M)~N \red Q$\\
$\Gamma \vdash N \in \csn$ \hfill by assumption \\
$\Gamma \vdash C_0[x]~M \in \csn$ \hfill by assumption \\
%
%$C[x]  \red R$ and $R = C_1[x]$ \hfill by Lemma \ref{lm:ecxt} \\
%$\Gamma \vdash C_1[x] \in \csn$ \hfill by previous lines \\
%$\Gamma \vdash C_1[x]~N \in \csn$ \hfill by IH(\ref{cp3}) \\
$\Gamma \vdash Q \in \csn$ \hfill by IH(\ref{cp3b})\\
$\Gamma \vdash C[x]\,N \in \csn$ \hfill by def. $\csn$
\\[1em]
\noindent
\fbox{\ref{cp3b}. 
If $\Gamma \vdash C[x]~M \red R$ and $\Gamma \vdash C[x] \in \csn$ 
and $\Gamma \vdash M \in \csn$ 
then $\Gamma \vdash R \in \csn$. }
\\[1em]
Proof by simultaneous induction on $\Gamma \vdash C[x] \in \csn$ and
$\Gamma \vdash M \in \csn$.
\\[0.5em]
\textbf{Sub-case} $\ianc{\Gamma \vdash C[x] \red M'}{\Gamma \vdash C[x]\,N \red M'\,N}{}$
\\[1em]
$\Gamma \vdash M' \in \csn$ \hfill by assumption $\Gamma \vdash C[x] \in \csn$ \\
$M'~N = C_0[x]$ \hfill by lemma \ref{lm:ecxt} \\
$M' = C_1[x]$ and hence $\Gamma \vdash C_1[x] \in \csn$ \hfill by def. evaluation contexts \\
$\Gamma \vdash N \in \csn$ \hfill by assumption \\
$\Gamma \vdash C_1[x]~N \in \csn$ \hfill by IH(\ref{cp3}) since
$\Gamma \vdash C_1[x] \in \csn$ is smaller than $\Gamma \vdash C[x] \in \csn$
\\[1em]
\\[1em]
\textbf{Sub-case} $\ianc{\Gamma \vdash N\red N'}{\Gamma \vdash C[x]\,N \red C[x]\,N'}{}$
\\[1em]
$\Gamma \vdash N' \in \csn$ \hfill by assumption $\Gamma \vdash N \in
\csn$ \\
$\Gamma \vdash C[x]~N' \in \csn$ \hfill by IH(\ref{cp3})
\\[1em]
\fbox{\ref{cp5}. If $\Gamma \vdash M\redsn M'$ and $\Gamma \vdash M' \in \csn$ then
     $\Gamma \vdash M \in \csn$. }
\\[1em]
Proof by induction on the first derivation.
\\
\paragraph{Case}$\D = \ianc{\Gamma \vdash N \in \csn}
                          {\Gamma \vdash (\lambda x.M)\;N \redsn [N/x]M}{}$
\\
$\Gamma \vdash [N/x]M \in \csn$ \hfill by assumption \\
$\Gamma \vdash (\lambda x.M)\;N \in \csn$ \hfill by Corollary \ref{cor:psn}(\ref{pp1})


\paragraph{Case} $\D = 
\ianc {\Gamma \vdash M \redsn M'} % \qquad M\,\text{is not a}\;\lambda
      {\Gamma \vdash M\;N \redsn M'\;N}{}$
\\
$\Gamma \vdash M'\;N \in \csn$ \hfill by assumption \\
$\Gamma \vdash M' \in \csn$ \hfill by Lemma \ref{lem:psn} (Property \ref{pp6})\\
$\Gamma \vdash M \in \csn$ \hfill by IH\\
$\Gamma \vdash N \in \csn$ \hfill by Lemma \ref{lem:psn} (Property \ref{pp6})\\
$\Gamma \vdash M~N \in \csn$ \hfill by Lemma \ref{lem:psn} (Property \ref{pp2})

% IH If $\Gamma \vdash M' \csn then $\Gamma \vdash M \csn$
%  H \Gamma \vdash M'\;N \in \csn
% --------------------------------------
%    \Gamma \vdash M\;N \in \csn


  % \begin{enumerate}
  % \item todo
  % \item Immediate by definition.
  % \item todo
  % \item By induction on the given derivation
  % \item todo
  % \end{enumerate}
\end{proof}


% Moreover, we have that if a given term $M$ is strongly normalizing, then any subterm must be strongly normalizing as well. We omit the proof for now and return to it later.\ednote{I don't have a good structural def of subterm, so not an easy exercise --am}. 

% \begin{theorem}[Subterm property of strong normalization]
% % Any subterm of a strongly normalizing term is strongly normalizing itself.za
% If $(\Gamma \vdash M) \in \csn$ and $\Gamma \vdash C[N] = M$ then $(\Gamma \vdash N) \in \csn$.
% % or more formally:
% %if $t \in \csn$ and $C[s] = t$ then $s \in \csn$.
% \end{theorem}

\subsection{Inductive Definition of Strongly Normalizing Terms}

Here, we define inductively the set of normal terms, $\SN$, and the set of neutral terms, $\SNe$,   using the following judgements:
\\[1em]

\begin{center}
\begin{tabular}{ll}
$(\Gamma \vdash M) \in \SN$  & $M$ is in the set of normal terms\\
$(\Gamma \vdash M) \in \SNe$ & $M$ is in the set of neutral terms 
\end{tabular}  
\end{center}

The inductive definition given in Fig.~\ref{fig:sn} is often more amendable for proofs than its informal definition, since it allows us  to prove properties by structural induction. 

\begin{figure}
  \centering  
\[
\begin{array}{c}
\multicolumn{1}{l}{\mbox{Neutral terms}} \\[1em]
\ianc{x{:}A \in \Gamma}{\Gamma \vdash x \in \SNe}{} \qquad  
\ibnc{\Gamma \vdash R \in \SNe}{\Gamma \vdash M \in \SN}{\Gamma \vdash R\,M \in \SNe}{} 
\\[1em]
\multicolumn{1}{l}{\mbox{Normal terms}} \\[1em]
\ianc{\Gamma \vdash R \in \SNe}{\Gamma \vdash R \in \SN}{} \qquad 
\ianc{\Gamma, x{:}A \vdash M \in \SN}{\Gamma \vdash \lambda x{:}A.M \in \SN}{} \qquad
\ibnc{\Gamma \vdash M \redSN M'}{\Gamma \vdash M' \in \SN}{\Gamma \vdash M \in \SN}{} 
\\[1em]
\multicolumn{1}{l}{\mbox{Strong head reduction}} \\[1em]
\ianc{\Gamma \vdash N \in \SN}{\Gamma \vdash (\lambda x.M)\;N \redSN [N/x]M}{} \qquad
\ianc{\Gamma \vdash R \redSN R'}{\Gamma \vdash R\,M \redSN R'\,M}{}
\end{array}
\]
  \caption{Inductive definition of strongly normalizing terms}
  \label{fig:sn}
\end{figure}

% \begin{lemma}[Variables are in $\SN$]
% If $x{:}A \in \Gamma$ then $\Gamma \vdash x \in SN$
% \end{lemma}
\begin{lemma}[Renaming]~\label{lm:renameSN}
  \begin{enumerate}
  \item If $\Gamma \vdash M \in \SN$ and $\Gamma' \ext{\rho} \Gamma$ then $\Gamma' \vdash [\rho]M \in \SN$  
  \item If $\Gamma \vdash M \in \SNe$ and $\Gamma' \ext{\rho} \Gamma$ then $\Gamma' \vdash [\rho]M \in \SNe$  
  \item If $\Gamma \vdash M \redSN N$ and $\Gamma' \ext{\rho} \Gamma$ then $\Gamma' \vdash [\rho]M \redSN [\rho]N$.
  \end{enumerate}
\end{lemma}
\begin{proof}
By induction on the first derivation.  

\paragraph{Case:} $\D = \ianc{\Gamma \vdash R \in \SNe}{\Gamma \vdash R \in \SN}{} $
\\[1em]
$\Gamma' \vdash [\rho]R \in \SNe$ \hfill by IH (2) \\
$\Gamma' \vdash [\rho]R \in \SN$ \hfill by def. of $\SN$

\paragraph{Case:} $\D = \ianc{\Gamma, x{:}A \vdash M \in \SN}{\Gamma \vdash \lambda x{:}A.M \in \SN}{}$
\\[1em]
$\Gamma', x{:}A \ext {\rho, x/x} \Gamma, x{:}A$ \hfill by def. of $\ext{\rho}$\\
$\Gamma', x{:}A \vdash [\rho, x/x]M \in \SN$ \hfill by IH (1) \\
$\Gamma' \vdash \lambda x{:}A.[\rho, x/x]M \in \SN$ \hfill by def. of $\SN$\\
$\Gamma' \vdash [\rho](\lambda x{:}A.M) \in \SN$ \hfill by subst. def.

\paragraph{Case:} $\D = \ibnc{\Gamma \vdash M \redSN M'}{\Gamma \vdash M' \in \SN}{\Gamma \vdash M \in \SN}{} $
\\[1em]
$\Gamma' \vdash [\rho]M \redSN [\rho]M'$ \hfill by IH (3) \\
$\Gamma' \vdash [\rho]M' \in \SN$ \hfill by IH (1)\\
$\Gamma' \vdash [\rho]M \in \SN$ \hfill by def. of $\SN$

\paragraph{Case:} $\D = \ianc{x{:}A \in \Gamma}{\Gamma \vdash x \in \SNe}{} $
\\[1em]
$\Gamma' \ext {\rho} \Gamma$ \hfill by assumption \\
$\Gamma' \vdash [\rho]x : A$ \hfill by renaming of typing \\
$\Gamma' \vdash [\rho]x \in \SNe$ \hfill by def. of $\SNe$

\paragraph{Case:} $\D = \ibnc{\Gamma \vdash R \in \SNe}{\Gamma \vdash M \in \SN}{\Gamma \vdash R\,M \in \SNe}{} $
\\[1em]
$\Gamma' \vdash [\rho]R \in \SNe$ \hfill by IH (2) \\
$\Gamma' \vdash [\rho]M \in \SN$ \hfill by IH (1) \\
$\Gamma' \vdash [\rho]R~[\rho]M \in \SNe$ \hfill by def. of $\SNe$\\
$\gamma' \vdash [\rho](R~M) \in \SNe$ \hfill by subst. def.

\paragraph{Case:}$\D = \ianc{\Gamma \vdash N \in \SN}{\Gamma \vdash (\lambda x.M)\;N \redSN [N/x]M}{} \qquad$
\\[1em]
$\Gamma' \vdash [\rho]N \in \SN$ \hfill by IH (1) \\
$\Gamma' \vdash [\rho](\lambda x.M)~[\rho]N \redSN [\rho, [\rho]N/x]M$ \hfill by def. of $\redSN$

\paragraph{Case:}$\D = \ianc{\Gamma \vdash R \redSN R'}{\Gamma \vdash R\,M \redSN R'\,M}{}$\\[1em]
$\Gamma' \vdash [\rho]R \redSN [\rho]R'$ \hfill by IH(3) \\
$\Gamma \vdash [\rho]R~[\rho]M \redSN [\rho]R'~[\rho]M$ \hfill by def. of $\redSN$\\
$\Gamma \vdash [\rho](R~M) \redSN [\rho](R'~M)$ \hfill by def. of subst.

\end{proof}

\begin{lemma}[Anti-Renaming]~\label{lm:anti-renameSN}
  \begin{enumerate}
  \item If $\Gamma' \vdash [\rho]M \in \SN$ and $\Gamma' \ext{\rho} \Gamma$ then $\Gamma \vdash M \in \SN$  
  \item If $\Gamma' \vdash [\rho]M \in \SNe$ and $\Gamma' \ext{\rho} \Gamma$ then $\Gamma \vdash M \in \SNe$  
  \item If $\Gamma' \vdash [\rho]M \redSN [\rho]N$ and $\Gamma' \ext{\rho} \Gamma$ then $\Gamma \vdash M \redSN N$.
  \end{enumerate}
\end{lemma}
\begin{proof}
By induction on the first derivation. [to check]  
\end{proof}

\begin{lemma}[$\SN$ is closed under application to variables]\label{lm:pSN1}
If $\Gamma \vdash M \in \SN$ and $x{:}A \in \Gamma$ then $\Gamma \vdash M~x \in \SN$.  
\end{lemma}
\begin{proof}
By induction on $\SN$.

\paragraph{Case:} $\D = \ianc{\Gamma \vdash R \in \SNe}{\Gamma \vdash R \in \SN}{}$
\\[1em]
$\Gamma \vdash x \in \SNe$ \hfill by def. of $\SNe$ \\
$\Gamma \vdash x \in \SN$ \hfill by def. of $\SN$ \\
$\Gamma \vdash R~x \in \SNe$ \hfill by def. of $\SNe$

\paragraph{Case:} $\D = \ianc{\Gamma, y{:}A \vdash M \in \SN}{\Gamma \vdash \lambda y{:}A.M \in \SN}{}$
\\[1em]
$\Gamma \ext{\id, x/y} \Gamma, y{:}A$ \hfill by def. of $\ext{\id}$ \\
$\Gamma \vdash [\id,x/y]M \in \SN$ \hfill by Lemma \ref{lm:renameSN} choosing $\Gamma \ext{\id, x/y} \Gamma, y{:}A$\\
$\Gamma \vdash x \in \SNe$  \hfill by def. of $\SNe$ \\
$\Gamma \vdash x \in \SN$  \hfill by def. of $\SN$ \\
$\Gamma \vdash (\lambda y{:}A.M)~x \redSN [\id,x/y]M$  \hfill by $\redSN$ \\
$\Gamma \vdash (\lambda y{:}A. M)~x \in \SN$\hfill by def. of $\SN$

\end{proof}


\begin{lemma}[Subterm Property of $\SN$]\label{lm:pSN2}
If $\Gamma \vdash M~N \in \SN$ then $\Gamma \vdash N \in \SN$.  
\end{lemma}
\begin{proof}
By Induction on $\SN$.
\paragraph{Case:} $\D =  \ianc{\Gamma \vdash R~N \in \SNe}{\Gamma \vdash R~N \in \SN}{}$  
\\[1em]
$\Gamma \vdash N \in \SN$ \hfill by def. of $\SNe$

\paragraph{Case:} $\D = \ibnc{\Gamma \vdash M~N \redSN Q}{\Gamma \vdash Q \in \SN}{\Gamma \vdash M~N \in \SN}{} $
\\[1em]
\textbf{Sub-Case:} $\D = \ianc{\Gamma \vdash N \in \SN}{\Gamma \vdash (\lambda x.M')\;N \redSN [N/x]M'}{}$ where $M = \lambda x.M'$\\[1em]
$\Gamma \vdash N \in \SN$ \hfill by assumption  
\\[1em]
\textbf{Sub-Case:} $\D = \ianc{\Gamma \vdash R \redSN R'}{\Gamma \vdash R\,M \redSN R'\,M}{}$ 
\\[1em]
$\Gamma \vdash R'\;M \in \SN$ \hfill by assumption since $Q = R'~M$\\
$\Gamma \vdash M \in \SN$ \hfill by IH

\end{proof}

% -- If t x �� SN then t �� SN.
We rely on the extensionality of $\SN$ for function types:

\begin{lemma}[Extentionality of $\SN$]\label{lm:pSN}
If $\Gamma \vdash M~x \in \SN$ then $\Gamma \vdash M \in \SN$.
\end{lemma}
\begin{proof}
By induction on $\SN$

\paragraph{Case:} $\D = \ianc{\Gamma \vdash M~x \in \SNe}{\Gamma \vdash M~x \in \SN}{} $  \\[1em]
 $\Gamma \vdash M \in \SNe$ \hfill by def. of $\SNe$ \\
 $\Gamma \vdash M \in \SN$ \hfill by def. of $\SN$
 \\[1em]
 \paragraph{Case:} $\D = \ibnc{\Gamma \vdash M~x \redSN Q}{\Gamma \vdash Q \in \SN}
                             {\Gamma \vdash M~x \in \SN}{} $
 \\[1em]
 \textbf{Sub-case}: $\Gamma \vdash (\lambda y.M')~x \redSN [x/y]M'$ \\[1em]
 $\Gamma \vdash [x/y]M' \in \SN$ \hfill by assumption \\
 $\Gamma, y{:}A \vdash M' \in \SN$ \hfill by Anti-Renaming Property (Lemma \ref{lm:anti-renameSN})\\
 $\Gamma \vdash \lambda y{:}A.M' \in \SN$ \hfill by def. of $\SN$
 \\[1em]
 \textbf{Sub-case}: $\Gamma \vdash M~x \redSN M'~x$ and $Q = M'~x$ \\[1em]
 $\Gamma \vdash M \redSN M'$ \hfill by def. of $\redSN$ \\
 $\Gamma \vdash M' \in \SN$ \hfill by IH \\
 $\Gamma \vdash M \in \SN$ \hfill by def. of $\SN$

% To Show: \Gamma \vdash M \in \SN$
\end{proof}

%We will sketch here that the inductive definition of $\SN$ and $\SNe$ is sound and complete with respect to our informal understanding of strongly normalizing reductions (Def. \ref{def:norm}). 

% We will write $M \in \csn$ for $M$ is strongly normalizing in our ``informal definition'', i.e. all reduction sequences starting in $M$ are finite, to distinguish it from our inductive definition in Figure \ref{fig:sn}. 




\subsection{Soundness and Completeness}
We can now prove that the two definitions of strongly normalizing terms coincide (soundness and completeness).

\begin{theorem}[Soundness of $\SN$]
\mbox{}
  \begin{enumerate}
  \item If $\Gamma \vdash M \in \SN$ then $\Gamma \vdash M \in \csn$.
  \item If $\Gamma \vdash C[x] \in \SNe$ then $\Gamma \vdash C[x] \in \csn$.
  \item If $\Gamma \vdash M \redSN M'$ then $\Gamma \vdash M \redsn M'$.
  \end{enumerate} 
\end{theorem}
\begin{proof}
By mutual structural induction on the given derivations using the
closure properties. \\[1em]
\noindent
1. If $\Gamma \vdash M \in \SN$ then $\Gamma \vdash M \in \csn$.

\paragraph{Case} $\D = \ianc{\Gamma \vdash R \in \SNe}{\Gamma \vdash R \in \SN}{}$ 
\\[1em]
$R = C[x] $ \hfill since $\Gamma \vdash R \in \SNe$ \\
$\Gamma \vdash R \in \csn$ \hfill by IH(1)

\paragraph{Case} $\D = \ianc{\Gamma, x{:}A \vdash M \in \SN}
                           {\Gamma  \vdash \lambda x{:}A.M \in \SN}{}$
\\[1em]
$\Gamma, x{:}A \vdash M \in \csn$ \hfill by IH(1) \\
$\Gamma \vdash \lambda x{:}A.M \in \csn$ \hfill by Closure Property \ref{lm:closn}(\ref{cp4})

\paragraph{Case} $\D = \ibnc{\Gamma \vdash M \redSN M'}{
                             \Gamma \vdash M' \in \SN}
                            {\Gamma \vdash M \in \SN}{} $
\\[1em]
$\Gamma \vdash M' \in \csn$ \hfill by IH(1)\\
$\Gamma \vdash M \redsn M'$ \hfill by IH(3)\\
$\Gamma \vdash M \in \csn$ \hfill by Closure Property \ref{lm:closn}(\ref{cp5})
\\[1em]
\noindent
2. If $\Gamma \vdash C[x] \in \SNe$ then $\Gamma \vdash C[x] \in \csn$.

\paragraph{Case} $\D = \ianc{x{:}A \in \Gamma}{\Gamma \vdash x \in \SNe}{}$
\\[1em]
$C = \_ $ \hfill since $C[x] = x$\\
$\forall M'.~\Gamma \vdash x \red M' \imply (\Gamma \vdash M') \in \csn$ \hfill since $\Gamma \vdash x \red M'$ is impossible
\\
$\Gamma \vdash x \in \csn$

\paragraph{Case}$\D = \ibnc{\Gamma \vdash R \in \SNe}{\Gamma \vdash M \in \SN}{\Gamma \vdash R\,M \in \SNe}{}$
\\
$C'[x] = R$ \hfill since $C[x] = R\;M$\\
$\Gamma \vdash C'[x] \in \csn$ \hfill by IH(2) \\
$\Gamma \vdash M \in \csn$ \hfill by IH(1)\\
$\Gamma \vdash C'[x]\;M \in \csn$ \hfill by Closure Property \ref{lm:closn}(\ref{cp3})\\
$\Gamma \vdash C[x] \in \csn$ \hfill since $C[x] = C'[x]~M$
\\[1em]
\noindent
3.  If $\Gamma \vdash M \redSN M'$ then $\Gamma \vdash M \redsn M'$.

\paragraph{Case} $\D = \ianc{\Gamma \vdash N \in \SN}{\Gamma \vdash (\lambda x.M)\;N \redSN [N/x]M}{}$
\\[1em]
$\Gamma \vdash N \in \csn$ \hfill by IH(1) \\
$\Gamma \vdash (\lambda x.M)\;N \redsn [N/x]M$ \hfill by def. of $\redsn$

\paragraph{Case} $\D = \ibnc{\Gamma \vdash R \redSN R'}{R\;\text{is not a}\;\lambda}{\Gamma \vdash R\,M \redSN R'\,M}{}$
\\[1em]
$\Gamma \vdash R \redsn R'$ \hfill by IH(3) \\
${\Gamma \vdash R\,M \redsn R'\,M}$ \hfill by def. or $\redsn$

\end{proof}



\begin{theorem}[Completeness of $\SN$]\mbox{}
\begin{enumerate}
\item\label{csn1} If $\Gamma \vdash C[x] \in \csn$ then $\Gamma \vdash C[x] \in \SNe$.
\item\label{csn2} If $\Gamma \vdash C[(\lambda x.M)~N] \in \csn$ 
      then $\Gamma \vdash  C[(\lambda x.M)~N] \redSN  C[[N/x]M]$. 
% \item If $\Gamma \vdash M \in \csn$ and $\Gamma \vdash M \redSN M'$
%  then $\Gamma \vdash M' \in \SN$.
 \item If $\Gamma \vdash M \in \csn$ then $\Gamma \vdash M \in \SN$.
%  \item If $R = x\,\vec N \in \csn$ then $x\,\vec N\in \SNe$.
%  \item If $R = (\lambda x.M)\,N\,\vec N \in \csn$ then $R \redSN [N/x]M\,\vec  N$.
%  \item If $R \in \csn$ then $R \in \SN$.
  \end{enumerate}  
\end{theorem}
\begin{proof}

\fbox{\ref{csn1}. If $\Gamma \vdash C[x] \in \csn$ then $\Gamma \vdash C[x] \in \SNe$.}
\\[1em]
By structural induction on $C[x]$.

\paragraph{Case} $C = \_$
\\[0.5em]
$\Gamma \vdash x \in \SNe$ \hfill by def. of
$\SNe$\footnote{Technically we should keep around explicitly that
  $\csn$ is defined on well-typed terms, so we actually have the
  assumption that $x{:}A$ is in $\Gamma$. }

\paragraph{Case} $C = C'~M$
\\[0.5em]
$\Gamma \vdash C'[x]~M \in \csn$ \hfill by assumption \\
$\Gamma \vdash C'[x] \in \csn$ and $\Gamma \vdash M \in \csn$ \hfill
by Lemma \ref{lem:psn} (Property \ref{pp6})\\
$\Gamma \vdash C'[x] \in \SNe$ \hfill by IH(1) \\
$\Gamma \vdash M \in \SN$ \hfill by IH (3) \\
$\Gamma \vdash C'[x]~M \in \SNe $ \hfill by def. of $\SNe$
\\[1em]
\fbox{\ref{csn2}. If $\Gamma \vdash C[(\lambda x.M)~N] \in \csn$ 
      then $\Gamma \vdash  C[(\lambda x.M)~N] \redSN  C[[N/x]M]$. }
\\[0.5em]
By structural induction on $C[(\lambda x.M)~N]$.

\paragraph{Case} $C = \_$
\\[0.5em]
$\Gamma \vdash (\lambda x.M)~N \in \csn$ \hfill by assumption \\
$\Gamma \vdash \lambda x.M \in \csn$ and $\Gamma \vdash N \in \csn$ \hfill by Lemma
\ref{lem:psn}(Property \ref{pp6})\\
$\Gamma \vdash N \in \SN$ \hfill by IH (3) \\
$\Gamma \vdash (\lambda x.M)~N \redSN [N/x]M$ \hfill by def. of
$\redSN$

\paragraph{Case} $C = C'~M'$
\\[0.5em]
$\Gamma \vdash C'[(\lambda x.M)~N]~M' \in \csn$ \hfill by assumption
\\
$\Gamma \vdash C'[(\lambda x.M)~N] \in \csn$ and $\Gamma \vdash N \in \csn$ \hfill by Lemma
\ref{lem:psn}(Property \ref{pp6})\\
$\Gamma \vdash  C'[(\lambda x.M)~N] \redSN  C'[[N/x]M]$ \hfill by IH (2)\\
$\Gamma \vdash  C'[(\lambda x.M)~N]~M' \redSN  C'[[N/x]M]~M'$ \hfill by def. $\redSN$
\\[1em]
\fbox{If $\Gamma \vdash M \in \csn$ then $\Gamma \vdash M \in \SN$.}
\\[1em]
Induction on $M$.

\paragraph{Case} $M = x$ where $x{:}A \in \Gamma$.
\\[1em]
$\Gamma \vdash x \in \SNe$ \hfill by def. of $\SNe$\\
$\Gamma \vdash x \in \SN$ \hfill by def. of $\SN$.

\paragraph{Case} $M = \lambda x.M'$
\\[1em]
$\Gamma \vdash \lambda x.M' \in \csn$ \hfill by assumption\\
$\Gamma, x{:}A \vdash M' \in \csn$ \hfill by Lemma \ref{lem:psn} (Property \ref{pp7})\\
$\Gamma, x{:}A \vdash M' \in \SN$ \hfill by IH \\ 
$\Gamma \vdash \lambda x{:}A.M' \in \SN$ \hfill  by def. of $\SN$ 

\paragraph{Case} $M = M_1~M_2$
\\[1em]
$\Gamma \vdash M_1~M_2 \in \csn$ \hfill by assumption\\
$\Gamma \vdash M_1 \in \csn$ and $\Gamma \vdash M_2 \in \csn$ \hfill by Lemma \ref{lem:psn} (Property \ref{pp6})
\\[0.5em]
 \textbf{Sub-case:} $C[x] = M_1~M_2$ \\[0.5em]
 $\Gamma \vdash M_1~M_2 \in \SNe$ \hfill by IH (1) (note this is valid,
 since (1) is strictly decreasing when we refer to (3))\\
 $\Gamma \vdash M_1~M_2 \in \SN$ \hfill by def. of $\SN$
 \\[1em]
 \textbf{Sub-case:} $C[(\lambda x.M)~N] = M_1~M_2$ \\[0.5em]
 $\Gamma \vdash C[(\lambda x.M)~N] \redSN C[[N/x]M]$ \hfill by IH(2)\\
$\Gamma \vdash C[[N/x]M] \in \SN$ \hfill ??\\
 % $\Gamma \vdash N \in \csn$ \hfill by generalization of by Lemma \ref{lem:psn} (Property \ref{pp6})\\
 % $\Gamma \vdash N \in \SN$ \hfil by IH(3) \\
 $\Gamma \vdash C[(\lambda x.M)~N] \in \SN$ \hfill by def of $\SN$ \\

\end{proof}


\section{Reducibility Candidates}
One might ask, what is a good definition of a semantic type? - Rather than
attempting the proof of the fundamental lemma directly and then trying to
extract additional lemmas one might need about the semantic types, we follow
Girard's technique and characterize some key properties our semantic types need
to satisfy. If a semantic type satisfies these key properties, then our proof of the fundamental lemma will be straightforward. To put it differently, defining these key properties, will allow for a  a modular proof of the fundamental lemma.

% \begin{definition}[Reducibility Candidate] $\Gamma \vdash M \in \den{A}$ is a reducibility
%   candidate, if the following conditions hold
%   \begin{itemize}
%   \item $\CR 1:$ If $\Gamma \vdash M \in \den{A}$ then $M \in \SN$. % , i.e. $\den{A} \subseteq \SN$.% \\[0.5em]
%   \item $\CR 2:$ If $\Gamma \vdash M \in \SNe$ then $\Gamma \vdash M \in \den{A}$. % , i.e. $\SNe \subseteq \den{A}$. % \\[0.5em]
%   \item $\CR 3:$ If $\Gamma \vdash M \redSN M'$ and $\Gamma \vdash M' \in \den{A}$ then $\Gamma \vdash M \in \den{A}$, i.e . $\den{A}$ is closed under reduction.
%   \end{itemize}  
% \end{definition}

% The last property is often also referred to as \emph{backward closed}. We show that that all semantic types $\den{A}$ satisfy the conditions above.



\begin{theorem}\label{thm:redcand}~
% For all types $C$, $\Gamma \vdash M \den{C}  \in \CR$, i.e. it satisfies the conditions $\CR_1$, $\CR_2$, and $\CR_3$.
  \begin{enumerate}
  \item\label{cr1} \CR 1: If $\Gamma \models M : A$ then $\Gamma \vdash M \in \SN$. % , i.e. $\den{A} \subseteq \SN$.% \\[0.5em]
  \item\label{cr2} \CR 2: If $\Gamma \vdash M \in \SNe$ then $\Gamma \models M : A$. % , i.e. $\SNe \subseteq \den{A}$. % \\[0.5em]
  \item\label{cr3} \CR 3: If $\Gamma \vdash M \redSN M'$ and $\Gamma \vdash M' \models A$ then $\Gamma \models M : A$, i.e. backwards closure.
  \end{enumerate} 
\end{theorem}
\begin{proof}
We prove these three properties simultaneously.
\\[1em]
\fbox{\CR \ref{cr1}.  If $\Gamma \models M : A$ then $\Gamma \vdash M \in \SN$.}
\\[0.5em]
By induction on the structure of $A$. 

\paragraph{Case: $C =\base$}.\\
$\Gamma \vdash M \models \base$ \hfill by assumption \\
$\Gamma \vdash M \in \SN$ \hfill by def. of sem. typing for $\base$

\paragraph{Case: $C = A \arrow B$}.
\\
$\Gamma, x{:}A \models x : A$ \hfill by Lemma \ref{lm:closn} (Property \ref{cp2})\\
$\Gamma, x{:}A \ext{\id} \Gamma$ \hfill by def. of context extensions \\
% $\Gamma, x{:}A \models M  : A \arrow B$ \hfill by Semantic Weakening Lemma \ref{lm:sweak}\\
$\Gamma, x{:}A \models M[\id]~x : B$ \hfill by def. of  $\Gamma, x{:}A \models M  : A \arrow B$\\
$\Gamma, x{:}A \vdash M[\id]~x\in \SN$ \hfill by IH (\CR \ref{cr1})\\
$\Gamma, x{:}A \vdash M[\id] \in \SN$ \hfill by  Extensionality Lemma \ref{lm:pSN} \\% Lemma \ref{lem:psn} (Property \ref{pp6})\\
$\Gamma \vdash M \in \SN$ \hfill by Ani-renaming Lemma \ref{lm:anti-renameSN}
\\[1em]
\fbox{\CR \ref{cr2}. If $\Gamma \vdash M \in \SNe$ then $\Gamma \models M : A$.}
\\[0.5em]
By induction on $\SNe$. 

\paragraph{Case: $C=\base$}.\\
$\Gamma \vdash M \in \SNe$ \hfill by assumption \\
$\Gamma \vdash M \in \SN$ \hfill by def. of $\SN$\\
$\Gamma \models M : \base$ \hfill by def. of semantic typing 

\paragraph{Case: $C = A \arrow B$}.\\
Assume $\Gamma' \ext{\rho} \Gamma$ and $\Gamma' \models N : A$ \\
$\Gamma' \vdash N \in \SN$ \hfill by IH (\CR \ref{cr1}) \\
$\Gamma \vdash M \in \SNe$ \hfill by assumption \\
$\Gamma' \vdash M[\rho] \in \SNe$ \hfill by weakening \\
$\Gamma' \vdash M[\rho]~N \in \SNe$ \hfill by def. of $\SNe$\\
$\Gamma' \models M[\rho]~N : B$ \hfill by IH (\CR \ref{cr2})\\
$\Gamma \models M : A \arrow B$ \hfill since $\Gamma' \models N : A$ was arbitrary
\\[1em]
\fbox{\CR \ref{cr3}.   If $\Gamma \vdash M \redSN M'$ and $\Gamma \vdash M' \models A$ then $\Gamma \models M : A$}
\\[0.5em]
By induction on $A$.
\paragraph{Case: $C = \base$}.\\[0.5em]
$\Gamma \vdash M' \in \SN$  \hfill since $\Gamma \vdash M' \models \base$\\
$\Gamma \vdash M \in \SN$ \hfill by closure rule for $\SN$\\
$\Gamma \models M : \base$ \hfill by definition of semantic typing 

\paragraph{Case: $C = A \arrow B$}.
\\[0.5em]
Assume $\Gamma' \ext{\rho} \Gamma$,~$\Gamma' \models N : A$ \\
$\Gamma' \models M'[\rho]~N : B$ \hfill by assumption $\Gamma \models M' : A \arrow B$\\
$\Gamma \vdash M \redSN M'$ \hfill by assumption \\
$\Gamma' \vdash M[\rho] \redSN M'[\rho]$ \hfill by Weakening \\
$\Gamma' \vdash M[\rho]~N \redSN M'[\rho]~N$ \hfill by $\redSN$\\
$\Gamma \models M[\rho]~N : B$ \hfill by IH (\CR\ref{cr3})\\
$\Gamma \models M : A \arrow B$ \hfill since $\Gamma' \models N : A$ was arbitrary\\

\end{proof}


\section{Proving strong normalization} 
As mentioned before, we prove that if a term is well-typed, then it is strongly normalizing in  two steps:

\begin{description}
\item[Step 1] If $\Gamma \models M : A$ then $\Gamma \vdash M \in \SN$. 
\item[Step 2] If $\Gamma \vdash M : A$ and $\Gamma' \models \sigma \Gamma$ then $\Gamma' \models M[\sigma] : A$.
\end{description}

The first part described in Step 1, is satisfied by the fact that $\Gamma \models M : A$ must be a reducibility candidate (Theorem \ref{thm:redcand}) and  by \CR \ref{cr1})  all terms in $\den{A}$ are strongly normalizing. We now prove the second step, which is often referred to as the \emph{Fundamental Lemma}.
It states that if $M$ has type $A$ and we can provide ``good'' instantiation $\sigma$, which provides terms which are themselves normalizing for all the free variables in $M$, then $\Gamma \models [\sigma]M : A$. 


\begin{lemma}[Fundamental lemma]
If $\Gamma \vdash M : A$ and $\Gamma' \models \sigma : \Gamma$
then $\Gamma' \models [\sigma]M : A$.  
\end{lemma}
\begin{proof}
By induction on $\Gamma \vdash M : A$.

\paragraph{Case} $\D = \ianc{\Gamma(x) = A}{\Gamma \vdash x : A}{}$
\\[1em]
$\Gamma' \models \sigma : \Gamma$ \hfill by assumption \\
$\Gamma' \models [\sigma]x : A$ \hfill by definition of $[\sigma]x$ and $\Gamma' \models \sigma : \Gamma$

\paragraph{Case} $\D = \ibnc{\Gamma \vdash M : A \rightarrow B}{\Gamma \vdash N : A}{\Gamma \vdash M\;N : B}{}$
\\
$\Gamma' \models \sigma : \Gamma$ \hfill by assumption \\
$\Gamma' \models [\sigma]M : A \rightarrow B $ \hfill by IH\\
$\Gamma' \models [\sigma]N : A$ \hfill by IH\\
$\Gamma' \models [\sigma]M\;[\sigma]N : B$ \hfill by $\Gamma' \models [\sigma]M : A \rightarrow B $\\
$\Gamma' \models [\sigma](M\;N) : B$ \hfill by subst. definition \\


\paragraph{Case} $\D = \ianc{\Gamma, x:A \vdash M:B}{\Gamma \vdash \lambda x.M : A \rightarrow B}{}$ 
\\
$\Gamma' \models \sigma : \Gamma$ \hfill by assumption \\
Assume $\Gamma'' \ext{\rho} \Gamma'$ and $\Gamma'' \vdash N : A$  \\
$\Gamma'' \models [\rho] \sigma: \Gamma$ \hfill by weakening \\
$\Gamma'' \models ([\rho]\sigma, N/x) : \Gamma, x:A$ \hfill by definition of semantic substitutions\\
$\Gamma'' \models [[\rho]\sigma, N/x]M : B$ \hfill by IH \\
$\Gamma'' \vdash (\lambda x.[[\rho]\sigma,x/x]M)\;N \redSN [[\rho]\sigma, N/x]M$ \hfill by reduction $\redSN$ \\
$(\lambda x.[[\rho]\sigma,x/x]M) = [[\rho]\sigma](\lambda x.M)$ \hfill by subst. def\\
$\Gamma'' \models ([[\rho]\sigma]\lambda x.M)\;N : B$ \hfill by $\CR 3$ \\
$\Gamma' \models [\sigma](\lambda x.M) : A \arrow B$ \hfill since $\Gamma'' \ext{\rho} \Gamma'$ and $\Gamma'' \vdash N : A$  was arbitrary

\end{proof}


\begin{corollary}
If $\Gamma \vdash M : A$ then $M \in \SN$.  
\end{corollary}

\begin{proof}
Using the fundamental lemma with the identity substitution $\Gamma \models \textsf{id} : \Gamma$, we obtain  $\Gamma \models M : A$. By $\CR 1$, we know $\Gamma \vdash M \in \SN$.
\end{proof}


\newpage
\renewcommand{\inl}{\textsf{inl}\;}
\renewcommand{\inr}{\textsf{inr}\;}
\newcommand{\caseof}[3]{\textsf{case}\,#1\,\textsf{of inl}\,x \Rightarrow #2 \mid \textsf{inr}\,y \Rightarrow #3}


\section{Extension: Disjoint sums}

We will now extend our simply-typed lambda-calculus to disjoint sums.

\[
\begin{array}{llcl}
\mbox{Types}  & A & \bnfas & \ldots \mid A + B\\
\mbox{Terms}  & M & \bnfas & \ldots \mid \inl M \mid \inr M \mid \caseof{M}{N_1}{N_2}
\end{array}
\]

Let us first extend our definition of $\SN$ and $\SNe$ (see Fig.~\ref{fig:sncase}).

\begin{figure}
 \centering
 
\[
\begin{array}{c}
\multicolumn{1}{l}{\mbox{Neutral terms}} \\% [0.5em]
\infer{\caseof{M}{N_1}{N_2} \in \SNe}{M \in \SNe & N_1 \in \SN & N_2 \in \SN}
\\% [0.5em]
%
\multicolumn{1}{l}{\mbox{Normal terms}} \\% [0.5em]
\infer{\inl M \in \SN}{M \in \SN} \qquad \infer{\inr M \in \SN}{M \in \SN} 
\\% [0.5em]
\multicolumn{1}{l}{\mbox{Strong head reduction}} \\[1em]
\infer{\caseof{(\inl M)}{N_1}{N_2} \redSN [M/x]N_1}{M \in \SN & N_2 \in \SN}
\\[0.75em]
\infer{\caseof{(\inr M)}{N_1}{N_2} \redSN [M/x]N_2}{M \in \SN & N_1 \in \SN}
\\[0.75em]
\infer{\caseof{M}{N_1}{N_2} \redSN \caseof{M'}{N_1}{N_2}}{M \redSN M'}
\end{array}
\]

   \caption{Inductive definition of strongly normalizing terms - extended for case-expressions and injections}
   \label{fig:sncase}
 \end{figure}


Next, we extend our definition of semantic type to disjoint sums. A first attempt might be to define $\den{A+B}$ as follows

\paragraph{Attempt 1}
\begin{center}
$\den{A+B} := \{\inl M \mid M \in \den{A} \} \cup \{\inr M \mid M \in \den{B} \}  $  
\end{center}

However, this definition would not satisfy the key property $\CR3$ and hence would fail to be a reducibility candidate. For example,  while $\inl y$ is in $\den{A + B}$, $(\lambda x. \inl x)\;y$ would not be in $\den{A+B}$ despite the fact that $(\lambda x. \inl x)\;y \redSN \inl y$. 
\\[1em]
Our definition of $\den{A+B}$ is not closed under the reduction relation $\redSN$. Let $\A$ denote the denotation of $\den{A}$. We then define the closure of $\den{A} = \A$, written as  $\clos\A$, inductively as follows:

\[
\begin{array}{c}
\ianc{M \in \A}{M\in \clos\A}{}  \qquad
\ianc{M \in \SNe}{M \in \clos\A}{} \qquad
\ibnc{M \in \clos\A}{N \redSN M}{N \in \clos\A}{}
\end{array}
\]

and we define 

\[
\begin{array}{lcl}
\den{A + B} & = & \clos{ \{\inl M \mid M \in \den{A} \} \cup \{\inr M \mid M \in \den{B} \}  }  
\end{array}
\]

\subsection{Semantic type $\den{A + B}$ is a reducibility candidate}
We first extend our previous theorem which states that all denotations of types must be in $\CR$.

\begin{theorem}
For all types $C$, $\den{C}  \in \CR$.
\end{theorem}
\begin{proof}
By induction on the structure of $A$. We highlight the case for disjoint sums.

\paragraph{Case $C = A + B$.} 

  \begin{enumerate}
  \item \textit{Show} $\CR1$. Assume that $M \in \den{A + B}$. We consider different subcases and prove by an induction on the closure defining $\den{A + B}$ that $M \in \SN$.

\paragraph{Subcase: $M \in \{\inl N \mid N \in \den{A}\}$}. Therefore $M = \inl N$. Since $N \in \den{A}$ and by i.h. ($\CR1$), $N \in \SN$. By definition of $\SN$, we have that $\inl N \in \SN$. 
  
\paragraph{Subcase: $M \in \{\inr N \mid N \in \den{B}\}$}. Therefore $M = \inr N$. Since $N \in \den{B}$ and by i.h. ($\CR1$), $N \in \SN$. By definition of $\SN$, we have that $\inr N \in \SN$. 

\paragraph{Subcase: $M \in \SNe$}. By definition of $\SN$, we conclude that $M \in \SN$.

\paragraph{Subcase: $M \in \den{A+B}$, if $M \redSN M'$ and $M' \in \den{A+B}$}.
\\[0.5em]
$M \redSN M'$ and $M' \in \den{A + B}$ \hfill by assumption\\
$M' \in \SN$ \hfill by inner i.h. \\ 
$M \in \SN$ \hfill by reduction $\redSN$

 \item \textit{Show} $\CR2$.if $M \in \SNe$, then $M \in \den{A + B}$\\
By definition of the closure, if $M \in \SNe$, we have $M \in \den{A + B}$.


  \item \textit{Show} $\CR3$. if $M \redSN M'$ and $M' \in \den{A+B}$ 
    then $M \in \den{A+B}$.\\
By definition of the closure, if $M \redSN M'$ and $M' \in \den{A+B}$, we have
$M \in \den{A+B}$.
\end{enumerate}

\end{proof}


 \subsection{Revisiting the fundamental lemma}

 We can now revisit the fundamental lemma.

 \begin{lemma}[Fundamental lemma]
 If $\Gamma \vdash M : A$ and $\sigma \in \den{\Gamma}$
 then $[\sigma]M \in \den{A}$.  
 \end{lemma}
 \begin{proof}
 By induction on $\Gamma \vdash M : A$.

 \paragraph{Case} $\D = \icnc{\Gamma \vdash M : A + B}{\Gamma, x{:}A \vdash N_1 :  C}{\Gamma, y{:}B \vdash N_2 : C}
 {\Gamma \vdash \caseof{M}{N_1}{N_2} : C}{}$
 \\[1em]
 $\sigma \in \den{\Gamma}$ \hfill by assumption \\
 $[\sigma]M \in \den{A + B}$ \hfill by i.h.
 \\[1em]
 We consider different subcases and prove by induction on the closure defining $\den{A + B}$, that $[\sigma](\caseof{M}{M_1}{M_2}) \in \den{C}$.

\paragraph{Subcase $[\sigma]M \in \{\inl N \mid N \in \den{A}\}$}$\;$\\[1em]
$[\sigma]M = \inl N$ for some $N \in \den{A}$ \hfill by assumption \\
$N \in \SN$ \hfill by $\CR1$ \\
% $x \in \den{A}$, 
$\sigma \in \den{\Gamma}$ \hfill by assumption \\
$[\sigma, N/x] \in \den{\Gamma, x:A}$ \hfill by definition \\
$[\sigma, N/x]M_1 \in \den{C}$ \hfill by outer i.h. \\
$y \in \den{B}$  \hfill by definition \\
$[\sigma, y/y] \in \den{\Gamma, y:B}$ \hfill by definition \\
$[\sigma, y/y]M_2 \in \den{C}$ \hfill by outer i.h. \\
$[\sigma, y/y]M_2 \in \SN$ \hfill by $\CR1$ \\
$\caseof{(\inl N)}{[\sigma,x/x]M_1}{[\sigma, y/y]M_2} \redSN [\sigma, N/x]M_1$ \hfill by $\redSN$\\
$\caseof{(\inl N)}{[\sigma,x/x]M_1}{[\sigma, y/y]M_2}$ \\
$\qquad = [\sigma](\caseof{M}{M_1}{M_2}) $ \hfill by subst. definition and $[\sigma]M = \inl N$\\
$[\sigma](\caseof{M}{M_1}{M_2}) \in \den{C}$ \hfill by $\CR3$

\paragraph{Subcase $[\sigma]M \in \{\inr N \mid N \in \den{B}\}$}$\;$\\[1em]
similar to the case above.

\paragraph{Subcase: $[\sigma]M \in \SNe$}.$\;$\\
$\sigma \in \Gamma$ \hfill by assumption \\
$x \in \den{A}$, $y \in \den{B}$  \hfill by definition \\
$[\sigma, y/y] \in \den{\Gamma, y:B}$ \hfill by definition \\
$[\sigma, x/x] \in \den{\Gamma, x:A}$ \hfill by definition \\
$[\sigma, x/x]M_1 \in \den{C}$ \hfill by outer i.h. \\
$[\sigma, y/y]M_2 \in \den{C}$ \hfill by outer i.h. \\
$[\sigma, y/y]M_2 \in \SN$ \hfill by $\CR1$ \\
$[\sigma, x/x]M_1 \in \SN$ \hfill by $\CR1$ \\
$\caseof{[\sigma]M}{[\sigma, x/x]M_1}{[\sigma, y/y]M_2} \in \SNe$ \hfill by $\SNe$ \\
$[\sigma](\caseof{M}{M_1}{M_2}) \in \SNe$ \hfill by substitution def. \\
$[\sigma](\caseof{M}{M_1}{M_2}) \in \den{C}$ \hfill by $\CR2$


\paragraph{Subcase:  $[\sigma]M \in \den{A + B}$, if $[\sigma]M \redSN M'$ and $M' \in \den{A+B}$}$\;$\\
$[\sigma]M \redSN M'$ and $M' \in \den{A+B}$ \hfill by assumption \\
$\caseof{M'}{[\sigma,x/x]M_1}{[\sigma,y/y]M_2} \in \den{C}$ \hfill by inner i.h. \\
$\caseof{[\sigma]M}{[\sigma,x/x]M_1}{[\sigma,y/y]M_2} $ \\
$\qquad\redSN
\caseof{M'}{[\sigma,x/x]M_1}{[\sigma,y/y]M_2}$ \hfill by $\redSN$\\
$[\sigma](\caseof{M}{M_1}{M_2}) \in \den{C}$ \hfill by $\CR3$
 \end{proof}


 \section{Extension: Recursion}
 \newcommand{\zero}{\mathsf{z}}
 % \renewcommand{\suc}{\textsf{s}\;}
 % \newcommand{\recnat}[3]{\textsf{rec} (#1,\,#2,\; #3)}
 \newcommand{\recnat}[3]{\recmatch{}{#1}{#2}{#3}}
 % \renewcommand{\nat}{\textsf{Nat}}
 We now extend our simply-typed lambda-calculus to include natural numbers
 defined by $\zero$ and $\suc t$ as well as a primitive recursion operator
 written as $\recnat{M}{M_z}{M_s}$ where $M$ is the argument we recurse over,
 $M_z$ describes the branch taken if $M = \zero$ and $M_s$ describes the branch
 taken when $M = \suc N$ where $n$ will be instantiated with $N$ and $f\;n$
 describes the recursive call.


\[
\begin{array}{llcl}
\mbox{Types}  & A & \bnfas & \ldots \mid \nat \\
\mbox{Terms}  & t & \bnfas & \ldots \mid \zero \mid \suc t \mid \recnat{t}{t_z}{t_s}
\end{array}
\]

To clarify, we give the typing rules for the additional constructs.

\[
\begin{array}{c}
\infer{\Gamma \vdash \zero : \nat}{} \qquad 
\infer{\Gamma \vdash \suc M : \nat}{\Gamma \vdash M : \nat}  
\\[1em]
\infer{\Gamma \vdash \recnat M {M_z} {M_s} : C}{
\Gamma \vdash M : \nat & \Gamma \vdash M_z : C & 
\Gamma, n:\nat,\;f\,n:C \vdash M_s : C}
\end{array}
\]


We again extend our definition of $\SN$ and $\SNe$.

\[
\begin{array}{c}
\multicolumn{1}{l}{\mbox{Neutral terms}} \\[1em]
\infer{\recnat{M}{M_z}{M_s} \in \SNe}{M \in \SNe & M_z \in \SN & M_s \in \SN}\\[1em]
\multicolumn{1}{l}{\mbox{Normal terms}} \\[1em]
\infer{\zero \in \SN}{} \qquad \infer{\suc M \in \SN}{M \in \SN}\\[1em]
\multicolumn{1}{l}{\mbox{Strong head reduction}} \\[1em]
\infer{\recnat{\zero}{M_z}{M_s} \redSN M_z}{M_s \in SN} \\[1em]
\infer{\recnat{(\suc N)}{M_z}{M_s} \redSN [N/n,\, f_r/f\,n]M_s}{
  N \in \SN & M_z \in \SN & M_s \in SN & f_r = \recnat{N}{M_z}{M_s}} \\[1em]
\infer{\recnat{M}{M_z}{M_s} \redSN \recnat{M'}{M_z}{M_s}}{M \redSN M'}
\end{array}
\]

 \section{Extension: Natural numbers}
Here we add natural numbers to our language and show how the language remains normalizing.
\subsection{Semantic type $\den{\nat}$} We define the denotation of $\nat$ as
 follows:

 \[
 \den{\nat} := \clos{\{\zero\}\cup \{\suc M \mid M \in \den{\nat}\} }
 \]


\subsection{Semantic type $\den{\nat}$ is a reducibility candidate}
We again extend our previous theorem which states that all denotations of types must be in $\CR$.

\begin{theorem}
For all types $C$, $\den{C}  \in \CR$.
\end{theorem}
\begin{proof}
By induction on the structure of $A$. We highlight the case for $\nat$.

\paragraph{Case $C = \nat$}

\begin{enumerate}
\item \textit{Show} $\CR1$: Assume $M \in \nat$. we consider different subcases
  and prove by induction on the closure defining $\nat$ that $M \in \SN$.
%

\paragraph{Subcase } $M = \zero$. By definition of $\SN$, $\zero \in \SN$.

\paragraph{Subcase } $M = \suc N$ where $N \in \den{\nat}$.  By i.h. ($\CR1$),
$N \in \SN$. By definition of $\SN$, $\suc N \in \SN$.

 \paragraph{Subcase} $M \in \SNe$. By definition of $\SN$, $M \in \SN$.

 \paragraph{Subcase} $M \in \den{\nat}$, if $M \redSN M'$ and $M' \in \den{\nat}$. 
 \\
 $M \redSN M'$ and $M' \in \den{\nat}$ \hfill by assumption \\
 $M' \in \SN$ \hfill by inner i.h. \\
 $M \in \SN$ \hfill by reduction $\redSN$
 \end{enumerate}


 \item \textit{Show} $\CR2$: By definition of the closure, $M \in \SNe$, then $M
   \in \den{\nat}$. 

 \item \textit{Show} $\CR3$: $M \in \nat$, if $M \redSN M'$ and $M' \in \nat$. By
   definition of the closure, we have that $M \in \nat$.
 \end{proof}

 \subsection{Revisiting the fundamental lemma}

 We can now revisit the fundamental lemma.
 \begin{lemma}[Fundamental lemma]
 If $\Gamma \vdash M : A$ and $\sigma \in \den{\Gamma}$
 then $[\sigma]M \in \den{A}$.  
 \end{lemma}
 \begin{proof}
 By induction on $\Gamma \vdash M : A$.

 \paragraph{Case} $\D = \ianc{}{\Gamma \vdash \zero : \nat}{}$
 \\
 $\zero \in \den{\nat}$ \hfill by definition.


 \paragraph{Case} $\D = \ianc{\Gamma \vdash M : \nat}{\Gamma \vdash \suc M :  \nat}{}$
 \\
 $\sigma \in \den{\Gamma}$ \hfill by assumption \\
 $M \in \den{\nat}$ \hfill by i.h. \\ 
 $\suc M \in \den{\nat}$ \hfill by definition 


 \paragraph{Case} $\D = \icnc
 {\Gamma \vdash M : \nat}
{\Gamma \vdash M_z : C}
 {\Gamma, n:\nat,\,f~n:C \vdash M_s : C}
 {\Gamma \vdash \recnat M {M_z} {M_s} : C}{}$
 \\
 $\sigma \in \den{\Gamma}$ \hfill by assumption \\
 $[\sigma]M \in \den{\nat}$ \hfill by i.h. \\[1em]
 We distinguish cases based on $M \in \den{\nat}$ and prove by induction on $M
 \in \den{\nat}$ that $[\sigma](\recnat M {M_z} {M_s}) \in \den{C}$.

 \paragraph{Subcase } $[\sigma]M = \zero$.
 \\
 $n \in \den{\nat}$, $f~n \in \den{C}$ \hfill by definition \\
 $[\sigma, n/n, f~n/f~n] \in \den{\Gamma, n:\nat, f~n:C}$ \hfill by definition \\
 $[\sigma, n/n, f~n/f~n]M_s \in \den{C}$ \hfill by outer i.h. \\
 $[\sigma, n/n, f~n/f~n]M_s \in \SN$ \hfill by $\CR1$ \\
 $[\sigma]M_z \in \den{C}$ \hfill by outer i.h. \\
 $\recnat \zero {[\sigma]M_z} {[\sigma, n/n,f~n/f~n]M_s} \redSN [\sigma]M_z$
 \hfill by $\redSN$ \\ 
$\recnat \zero {[\sigma]M_z} {[\sigma, n/n,f~n/f~n]M_s} = [\sigma](\recnat{M}
 {M_z} {M_s})$ \hfill by subst. def. and $M = \zero$\\
 $[\sigma](\recnat{M} {M_z}{M_s} \in \den{C}$ \hfill by $\CR3$.

 \paragraph{Subcase } $[\sigma]M = \suc M'$ where $M' \in \den{\nat}$.
 \\
 $M' \in \den{\nat}$ \hfill by assumption \\
 $M' \in \SN$ \hfill by $\CR1$ \\
 $[\sigma]M_z \in \den{C}$ \hfill by outer i.h. \\
 $[\sigma]M_z \in \SN$ \hfill by $\CR1$ \\
 $[\sigma, n/n, f~n/f~n]M_s \in \den{C}$ \hfill by outer i.h. \\
$[\sigma, n/n, f~n/f~n]M_s \in \SN$ \hfill by $\CR1$ \\
 $\recnat{M'}{[\sigma]M_z}{[\sigma, n/n, f~n/f~n]M_s} \in \den{C}$ \hfill by inner i.h. \\
 $[\sigma, M'/x, \;\recnat{M'}{[\sigma]M_z}{[\sigma, n/n, f~n/f~n]M_s}/f~n] \in
 \den{\Gamma, n:\nat, f~n:C}$ \hfill by definition \\
 $[\sigma, M'/x, \;\recnat{M'}{[\sigma]M_z}{[\sigma, n/n, f~n/f~n]M_s}/f~n]M_s \in \den{C}$ \hfill by outer i.h. \\
 $\recnat{(\suc M')}{[\sigma]M_z}{[\sigma, n/n, f~n/f~n]M_s} $ \\
 $\qquad \redSN
 [\sigma, M'/x, \;\recnat{M'}{[\sigma]M_z}{[\sigma, n/n, f~n/f~n]M_s}/f~n]M_s$
 \hfill by $\redSN$ \\
 $[\sigma](\recnat{M}{M_z}{M_s}) \in \den{C}$ \hfill by $\CR3$.

 \paragraph{Subcase } $[\sigma]M \in \SNe$. \\
 $[\sigma]M_z \in \den{C}$ \hfill by outer i.h.\\
 $[\sigma]M_z \in \SN$ \hfill by $\CR1$\\
 $[\sigma, n/n, f~n/f~n]M_s \in \den{C}$ \hfill by outer i.h. \\
 $[\sigma, n/n, f~n/f~n]M_s \in \SN$ \hfill by $\CR1$ \\
 $\recnat{[\sigma]M}{[\sigma]M_z}{[\sigma, n/n, f~n/f~n]M_s} \in \SNe$ \hfill by $\SNe$\\
$[\sigma](\recnat{M}{M_z}{M_s}) \in \SNe$ \hfill by subst. def. \\
$[\sigma](\recnat{M}{M_z}{M_s}) \in \den{C}$ \hfill by $\CR2$.


 \paragraph{Subcase } $[\sigma]M \in \den{\nat}$, if $[\sigma]M \redSN M'$ and
 $M' \in \den{\nat}$.\\
 $[\sigma]M \redSN M'$ and $M' \in \den{\nat}$ \hfill by assumption.\\
 $\recnat{M'}{[\sigma]M_z}{[\sigma,n/n, f~n/f~n]M_s} \in \den{C}$ \hfill by inner i.h. \\ 
 $\recnat{[\sigma]M}{[\sigma]M_z}{[\sigma,n/n, f~n/f~n]M_s}$ \\
 $\qquad \redSN \recnat{M'}{[\sigma]M_z}{[\sigma,n/n, f~n/f~n]M_s}$ \hfill by $\redSN$\\
 $[\sigma](\recnat{M}{M_z}{M_s}) \in \den{C}$ \hfill by $\CR3$.


 \end{proof}


%  \section{Exercises}
%  \begin{problem}
% % \begin{exercise}
%  The Def. \ref{def:norm}, defines strong normalization informally. We can replace this definition with a more formal definition.

%  \begin{definition}[Inductive definition of strongly normalizing terms] A term $M$ is strongly normalizing, if all its reducts are strongly normalizing, i.e. 
%  $M \csn$ if for all $M'$, if $M \red M'$ then $M' \in \csn$.  
%  \end{definition}

%  This definition gives rise to an induction principle to reason about strongly normalizing terms. To prove $\forall M \csn. P(M)$, we can assume the property holds for $P(M')$ for any $M'$ s.t. $M \red M'$. Using this induction principle, we can now prove  constructively that any subterm of a strongly normalizing term is itself normalizing.

%  Before we however, precisely define our notion of subterm to simplify our reasoning. We write $M = C[N]$ for $N$ is a subterm of $M$; $C$ denotes an evaluation context, i.e. the term $M$ where we identify $N$ as a subterm at a given position. Evaluation contexts can  be defined as follows.

%  \[
%  \begin{array}{lcl}
%  \mbox{Evaluation Context}\;C & \bnfas & [ ] \mid \lambda x. C \mid C\;N \mid M\;C
%  \end{array}
%  \]

%  As an alternative to the congruence rules, we can redefine evaluation using evaluation contexts:

%  \[
%  \ianc{M \red N}{C[M] \red C[N]}{}
%  \]

% \begin{theorem}
% Any subterm of a strongly normalizing term is strongly normalizing itself, i.e. if $M \csn$ and $N$ is a subterm of $M$, i.e. $M = C[N]$ then $N \in \csn$.
% \end{theorem}
% \end{exercise}  
% \end{problem}



% \begin{problem}
%   \begin{exercise}
% Extend the semantic strong normalization proof to treat $A \times B$. 
% \begin{itemize}
% \item Extend our definition of normal and neutral terms (i.e. $\SN$ and $\SNe$). You might also need to extend our definition of strong head reduction (i.e. $\redSN$).
% \item Define an appropriate denotation of $\den{A \times B}$.
% % \den{A * B} := {M | \fst(M) \in \den{A} or \snd{M} \in \den{B}}
% \item Show that $\den{A \times B}$ is a reducibility candidate.
% \item Show the additional cases in the fundamental lemma.
% \end{itemize}    
%   \end{exercise}
% \end{problem}


%\section{Reducibility candidates}
%\section{Forward and backward closed}
%\section{Fundamental lemma}
\end{document}
%%% Local Variables: 
%%% mode: latex
%%% TeX-master: "sn-proof"
%%% End: 
